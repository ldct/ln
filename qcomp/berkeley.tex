\listfiles
\documentclass{article}

\usepackage[pdftex]{graphicx}
\usepackage{amsmath}
\usepackage{amssymb}

\usepackage[a4paper,margin=1in]{geometry}
\setlength\parindent{0pt}
\setlength\parskip{12pt}

\newcommand{\half}{\frac{1}{2}}
\newcommand{\<}{\langle}
\renewcommand{\>}{\rangle}
\newcommand{\q}[1] {|{#1 \rangle}}


\title{}
\date{}

\begin{document}
\maketitle


\section{Double slit experiment}

Simple experiment that ``captures" quantumn weirdness. Set up a source of particles (electrons, photons etc) and a detector screen. Between them, place a blocking material with only two slits.

\begin{itemize}
\item When slits are observed: two humps
\item When slits are not observed: interference pattern
\end{itemize}

Clasically, interference only for waves.

Strange features of the double slit experiment:

\begin{enumerate}
\item Probabilistic
\item Observe $\implies$ interfere
\end{enumerate}

\section{Qubit Model}

In quantumn mechanics, many observables are quantized (forced to have certain discrete values). For example electrons have quantized energy states.

\subsection{Superposition Axiom}

If there exist $k$ classical states

$$
\q{0}, \q{1} \cdots \q{k-1}
$$

There there also exists a quantum superposition state

$$
\alpha_0 \q{0} + \alpha_1 \q{1} \cdots \alpha_{k-1} \q{k-1}
$$

Where $\alpha_i \in \mathbb{C}, \sum|\alpha_i|^2 = 1$

\subsection{Measurement Axiom}

When measuring in the classical state basis, we observe $j$ with probabiliity $|a_j|^2$ and the new state is $\q{j}$.

\subsection{Qubit}

$$
E < E_2
$$

Then it can only occupy two states. We assign ground state $\to \q{0}$, excited state $\to \q{1}$

Then by Schrodinger's equation

$$
\psi(t) \sim \cos\left(\frac{\Delta E}{\hbar} t\right)
$$

Use light of frequency $\nu = \frac{\hbar}{\Delta E} \approx 2.5 \times 10^{15} Hz$ for most atoms to manipulate the qubit.

\subsection{Geometrical Interpretation}

The qubit $\gamma = (\alpha_0, \alpha_1 \cdots \alpha_{k-1})$ is a vector in Hilbert space.

\subsection{Arbitrary Basis}

We can define the sign basis

$$
\q{\pm} = \q{0} \pm \q{1}
$$

and measurements with respect to it

$$
Pr[+] = |\< \psi, + \>|^2
$$

\subsection{Uncertainty}

If

\begin{align}
\q{\psi} &= \alpha_0 \q{0} + \alpha_1 \q{1} \\
&= \beta_+ \q{+} + \beta_- \q{-}
\end{align}

Let

\begin{align}
S\psi &= |\alpha_0| + |\alpha_1| \\
\tilde{S}\psi &= |\beta_+| + |\beta_-|
\end{align}

$S$ mesaures spread in $\q{0}, \q{1}$ basis, $\tilde{S}$ in $\q{\pm}$ basis; for example,

$$
\begin{array}{ll}
S \q{0} = 1 & S \q{+} = \sqrt 2 \\
\tilde{S} \q{0} = \sqrt{2} & \tilde{S} \q{+} = 1
\end{array}
$$

We have, for any $\q{\psi}$,

$$
S\q{\psi}\tilde{S}\q{\psi} \ge \sqrt{2}
$$

\section{Multiple Qubits}

Consider two qubits. Classical states: $00$, $01$, $10$, $11$ - two bits of information. By the superposition axiom, the quantumn state is

$$
\q{\psi} = \alpha_{00} \q{00} + \alpha_{01} \q{01} + \alpha_{10} \q{10} + \alpha_{11} \q{11}
$$

what happens if we measure just one qubit? Then

\begin{align}
0 \to \alpha_{00} \q{00} + \alpha_{01} \q{01} \\
1 \to \alpha_{10} \q{10} + \alpha_{11} \q{11}
\end{align}

where the RHS in unormalized, with probabilities $|\alpha_{00}|^2 + |\alpha_{01}|^2$ and $|\alpha_{10}|^2 + |\alpha_{11}|^2$ respectively.

\subsection{Factorization}

Say there are two qubits with states $\alpha_0 \q{0} + \alpha_1 \q{1}$ and $\beta_0 \q{0} + \beta_1 \q{1}$ respectively. Composite system has state

$$
(\alpha_0 \q{0} + \alpha_1 \q{1}) \otimes (\beta_0 \q{0} + \beta_1 \q{1}) = \alpha_0 \beta_0 \q{00} + \alpha_0 \beta_1 \q{01} + \alpha_1 \beta_0 \q{10} + \beta_1 \alpha_1 \q{11}
$$

\subsection{Entanglement}

Can you always factorize? No! Example:

$$
\q{\Phi^+} = \frac{1}{\sqrt 2} (\q{00} + \q{11})
$$

Measure first qubit

\begin{align}
0 \to \q{00} \\
1 \to \q{11}
\end{align}

Perfectly correlated; it's as though they flipped a coin to decide result of $0/1$ measurement (bit measurement) when they were together.

\subsection{EPR Paradox}

\begin{align}
\q{\Phi^+} &= \frac{1}{\sqrt 2} (\q{00} + \q{11}) \\
&= \frac{1}{\sqrt 2}(\q{++} + \q{--})
\end{align}

If they flipped a coin for both mit and sign measurement, this violates the uncertainty principle! Eg: measure first qubit in bit basis and second in sign basis. Now we know bit and sign with no uncertainty!

Resolution: They did not flip a coin. As soon as any bit is measured, entanglement is destroyed. There is no FTL travel.

\subsection{Bell's Experiment}

Local Hidden Variable Theory: both paricles carry all the information needed to decide outcome of any measurement.

Imagine Alice and Bob receive random bits $x$ and $y$ and need to try to minimize

$$
P[xy = a \oplus b]
$$

Under a LHVT, $ P \le 0.75$; however, with QM, they can get up to 0.85! Here's how...

\subsection{Rotational Invariance}


\begin{align}
\q{\Phi^+} &= \frac{1}{\sqrt 2} (\q{00} + \q{11}) \\
&= \frac{1}{\sqrt 2} (\q{uu} + \q{u^\perp u^\perp})
\end{align}

If we measure the qubits in different bases, $P[match] = \cos^2\theta$

\subsection{Bell's choice of angles}

$$
\begin{array}{cc}
Alice & Bob \\
0: 0 & 0: \frac{\pi}{8}\\
1: \frac{\pi}{4} & 1: -\frac{\pi}{8}
\end{array}
$$

Visually,

$$
-y_1--x_0--y_0--x_1-
$$

Now when $xy = 0$, $\theta = \frac{\pi}{8}$, $P[correct] = P[match] = \cos^2\frac{\pi}{8}$

When $xy = 1$, $\theta = \frac{3\pi}{8}$, $P[correct] = 1 - P[match] = 1 - \cos^2\frac{3\pi}{8} = \cos^2\frac{\pi}{8}$

QED

\subsection{Certifying Randomness}

Quantumn $\implies$ truly random

$P \approx 0.85 \implies$ quantum

\section{Quantumn Gates}

Qubits evolve through unitary evolution. Angle-preserving, rigid-body.

$$
U U^\dagger = I
$$

Linear operator:

$$
U(\alpha\q{0} + \beta\q{1}) = \alpha U\q{0} + \beta U\q{1}
$$

\subsection{Single qubit gates}

Bit flip
Phase flip
Hadamard

\subsection{No cloning theorem}

No cloning allowed!

\subsection{Two qubit gates}

$4 \times 4$ unitary matrix over $\mathbb{C}^2 \times \mathbb{C}^2 = \mathbb{C}^4$

To descibe two qubit gate, describe its action on $\q{00}, \q{01}$ etc, then use linearity

\subsection{CNOT Gate}

\begin{align}
a -&- a \\
b -&- a \oplus b
\end{align}

$$
\left(
\begin{array}{cccc}
1 & 0\\
0 & 1 \\
& & 0 & 1 \\
& & 1 & 0
\end{array}
\right)
$$

\subsection{Preparing Bell state}

\end{document}
