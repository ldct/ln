\listfiles
\documentclass{article}

\usepackage[pdftex]{graphicx}
\usepackage{amsmath}
\usepackage{amssymb}

\usepackage[a4paper,margin=1in]{geometry}
\setlength\parindent{0pt}
\setlength\parskip{12pt}

\newcommand{\half}{\frac{1}{2}}
\newcommand{\<}{\langle}
\renewcommand{\>}{\rangle}

\title{}
\date{}

\begin{document}
\maketitle

\section{Double slit experiment}


So quantum mechanics is a very strange theory. Nothing you have studied about in the classical physics, nothing in the experience you have with the physical world, can prepare you for it. It's just the case that, at the level of elementary particles, nature behaves in a very strange way. And this strange way is described by this theory called quantum mechanics. In fact this theory is so strange that Feynman once said, nobody understands quantum mechanics - by which he meant, not that nobody understands it, but that nobody intuitively understands why nature behaves the way it does at the quantum level. 

Okay, so let me describe some of these funny aspects of quantum mechanics. So the first aspect of quantum mechanics that's strange is that it's inherently probabilistic. You know when you measure something about an elementary particle, the result is always sample from a probability distribution. It's not a determined quantity; you only get probabilistic information about the system. The second aspect of quantum mechanics which is bizarre is that you cannot make a measurement of a system without disturbing it. And then there are many other funny aspects of quantum mechanics so elementary particles like electrons and photons behave like nothing at all that you're used to in the classical world so they behave neither like particles, nor like waves, even thought they share some of the features of each of them. What's interesting, though, is that all these particles so photons, electrons, etc, they all behave in the same way. Not like particles, not like waves, but like something else and it's always the same something. 

So in this lecture I will describe a particular experiment called the double-slit experiment, which highlights both the commonality as well as the differences between the behavior of quantum particles. So we'll be able to see in what sense quantum mechanics gives us particle like behavior and what sense it gives us wave like behavior and what sense it deviates from both of these. Now this experiment is a very basic experiment but it illustrates many of the fundamental features of quantum mechanics. And starting in the next lecture, I'll start with by describing qubits, or quantum bits and we'll do an entirely self contained exploration of basic quantum mechanics in terms of quantum bits. 

In the double-slit experiment we have a source of either light from a photons or electrons. And then there's a screen with a single small slit in it through which these electrons or photons have to pass. At some distance there's another screen with two slits carved into it. We'll label these slits slit one and slit two. And then a long way from there, there's another screen, you know which is where we are going to detect where these electrons or photons end up. 

Let's try to understand what the behavior of this experiment would be if the source was shooting classical particles which will model as bullets. So let's think of this source as a machine-gun which is firing bullets, you know bullets because they are discrete objects,they are indestructible. We think of them as discrete indestructible objects and so this slit then becomes an armor plate with a, with a hole in it and the detector is just a sandbox. Which we place at a distance $x$ from some fixed reference point. We'll also assume that this machine gun is sort of jiggly, it's not held very firm so that the bullets spray along different trajectories. And moreover I'll assume that this slit here is narrow enough that the bullets actually tend to strike the edge and spray off the edge so that if the bullets happen to go through this hole, they actually spray according to some angle out here. And then some of these bullets then make it to one of these slits, slit one or slit 2. 

So now of course, you can keep the machine gun firing and atsome rate and you can see how many bullets end up at this detector at location $x$. Of course, this is going to be a random variable because as we said, the bullets ricochet off at random angles. So you can only talk about how many bullets end up in this detector over a certain period of time. Let's say that on average at $x$ you have some number of bullets. Let's say three, three per minute so; we'll think of that as the average rate at which bullets end up at x. Now of course this three does not have to be a natural number; it can also be some number like 3.5 or 3.1 And what thatmeans is, on average in the cost of an, of a minute you get 3.1 bullets. Meaning, ifyou actually did the measurement over a period of ten minutes and you got 31bullets during that time, you would say that the rate at which bullets are endingup in this detector is 3.1 per minute. And so now you can plot how many bullets end up in this detector as a function of x? And you might get some curve like this 

so it might have two humps in it, two peaks corresponding to straight line parts going through slit one or slit two and it falls off on either end. Now you can also close one of these slits and ask how many bullets end up if you only have slit one of them, or how many end up if you only have slit two of them? And you get these two different curves, curves like this. And so if you label this curve as $n1 of x and this curve has n2 of x, that's good. Then,when both slit one and slit two are open, we get this curve n12 and the equationthat it satisfies is that n12 is equal to n1 + n2. So the number of bullets that endup at x if both slits are open is the number that ends up if only slit, slit onewas open, plus a number if only slit two was open, makes a lot of sense. That's,that's very simple so now let's, let's repeat this experiment and now let's,let's imagine that we are, we are doing this experiment with water waves, withwaves. So what we have is a little pond. The source is some vibrating object whichsets up these little waves and let's say the vibrating object is vibrating at, at afixed frequency so that the waves are nice and steady. And, you have some sort of abarrier here with a little slit through it so that, so that as the, as the wave getsthrough this, this little slit it starts spreading. And then it spreads all the wayuntil it gets to the second barrier with the two slits. And again, the wave startsspreading through the two slits. And now our detector is just going to detectwhat's the intensity of the, of the disturbance at point x. What's the energyof the wave at point x? So one way to measure it is to put some object like acork in the water, and, and you see how vigorously it gets disturbed. What's theenergy of the cork? And that's what we're going to measure. And so when we measureit as a function of x, that intensity which let's say we call that I12. Denotingthe fact that as a function of x. Denoting the fact that both slits one and slit twoare open. It follows this very funny curve which, which you will probably recognizefrom maybe high school physics as the interference pattern. Okay so, so what,what happens in this interference pattern? So remember we said that we have a wavecoming through each of the two slits. And now if you look at the point right in themiddle here where you have this, this crest, this big crest, right? So whathappens at this point in the detector? What happens is, since it is equidistantfrom both the slits and since the two waves from the two slits are completely insync, crests from both the waves appear simultaneously here because they have anequal distance to travel from here to here. And troughs appear, alsosimultaneously so that you have constructive interference between thecrests and constructive interference between the, between the troughs. And so,the crests become higher and the troughs become lower. And so the, energy of this,you know the disturbance here, is really much higher, you know, the waves reinforceeach other. On the other hand, if you move a little further away from the center,then about the time the, the waves crest arrives from slit two is the time thatthe, that the j - first trough arrives from slit one because they're off byexactly half a wavelength. And so they more or less cancel each other out. Nowperhaps, perhaps the wave from slit one, is, is a little bit stronger than the onefrom slit two here. And so they don't quite cancel out, but they very nearlycancel out. And so you get nearly completely destructive interference and,and the water is essentially close to being still. And so the cup of coffee isimparted almost no energy at all. And similarly if you, if you move a littlefurther and again you get constructive interference here because, because nowwhen the, when the fourth crest arrives from slit two, it's at the same time thatthe third crest arrives from slit one and you get constructive interference again.So that, that accounts for this sort of, you know this sort of pattern. On theother hand, if you were to open only slit one, and ask, what's the intensity as afunction of x? You'll get this, you know this curve that we saw with bullets, I1(x)and with slit two you get, you get this curve I2(x). And, I12(x) is not equal toI1(x) + I2(x). Okay, so how do you get I12 from I1 and I2? Well, it's actually verysimple. So the intensity at x, the intensity or the energy of the wave isjust the square, is proportional to the square of the height of the wave. And so,what you have is that the height, when both slits are open is exactly equal tothe height when only the first slit is open plus the height when the second slitis open. But of course, what this means is that I12(x) which is H12(x) the wholething squared is not equal to H1(x)^2 + H2(x)^2. In particular, if H1 is verynearly equal to -H2 then H12 is zero, very close to zero. So, I12 is very close tozero even though. I1 and I2 are each not close to zero. Okay, so that's the casewith, with waves.$


Now let's look at the situation when we have a source of electrons or photons. So, let's say it's a source of electrons. It's an electron gun and again we have, you know we have the same sort of set up with the detector in the back here which we can think of as, you know the screen in the back is fluorescent so that whenever an electron hits you get, you get little burst of light. And now, we can ask what's the intensity of the electrons arriving at, at point x? How much light do we see? How many do we detect? And the, the thing that happens here is that, well, one thing that you noticed is as you turn down the intensity of the source of electrons, you stop noticing that the electrons start arriving at point x at discreet points in time. So, you see a flash and then nothing for a while and then another flash, nothing for a while and so on. And as you turn down the intensity, the flashes don't get any, any less intense. What changes is the frequency with what you see with the flashes. So, again what we have to say is that, you know that would seem to tell us that these electrons are really particles. You know what they are, they are particles with, you know these charge particles and, and you know as you turn down the electron, intensity of the electron source. The fewer and fewer electrons going out per unit time and as they go out and they are not through you know they're deflected through the edges of these slits. They randomly show up at some point x. And the, the probability that you see in electron here depends upon, you know, you turned down the intensity of the source you see them less and less frequently but they arrive as discreet objects as discreet lumps. And so, you can talk about ah, the probability of detecting the, the electron at point x, you know, you can call it as intensity i(x). And so, now, we would, you know, given that, given that they are like discreet particles or bullets, what kind of behavior would be expect. And so, again, if he, if he open only slit one, the intensity as a function of x looks like this. If you open only slit two, the intensity of the function of x. The probability that we see, see in electron looks like this. And, since, since electrons are behaving like particles like bullets, you would imagine that if both slits are open, you should see this curve which is i12 which is sum of i1 and i2. But in fact, what you end up seeing is this interference pattern like we did in the case of [inaudible]. I12 is not equal to i1. + i2. And that's the strange thing about quantum mechanics. So how could it be if electrons are traveling, if they are, if they are like particles, if they are discrete objects, discrete you know, indestructible objects, how could it be that when both slits are open, you do not get to see the sum of these two curves as the probability of, of the electron ending up at x. You could say, well, let's reason about this a little more carefully and say, well, clearly the electron was fired through the source. It went through the source, it called deflected. And then it either went through slit one or through slit two. And if it went through slit one, it ended up at x with probability equal to i1(x). If it went through slit two it ended up at x with probability i2(x). Surely, if both slits were open. It should end up with x with probability i1 + i2(x). Because after all if it went through slit one, why should it matter to it whether slit two was open or not. And now, the answer is we don't know but, but when you do the experiment, you get to see the interference battle. Now, in quantum mechanics, we have a way of explaining this. What we can do is we can say, well, actually, there's an amplitude which, which the electrons gets up at one and ends up at x. And that amplitude is a1(x). And actually the, the, the probability that we detect the proton at point x. I(x) is actually the square of a1(x). And similarly, there's an amplitude with which it goes through slit two and ends up at x. And if only slit two is open, i2(x) is just the square of this amplitude. And similarly, if both slits are open, then a1,2 is a1. + a2. So that the amplitude with which the photon ends up at x is just a1(x) + a2(x). And of course the probability that you've detect before turn down is a12(x) squared. This is just like the [inaudible] case where we have the height of the water wave and the intensity is the square of the height of the water wave except that there's no height here. So what is this amplitude? Well, we don't know but this is how nature behaves. The electrons behave as to there was some amplitude with which it ends up at x and this amplitude can be positive or negative leading to this kind of interference battle. That's the funny thing about quantum mechanics. That's how electrons and protons behave. So, let's summarize what we've learned. So, we did this double slit experiment three times in three different settings. First, we considered it. It was a source of particles or bullets which we think of as bullets. Then we repeated this experiment with waves, with water waves. And finally, we repeated it with quantum objects like with elementary particles like photons, electrons. So, of course, in the case of bullets, we have discreet objects that come, you know, bullets come as discreet chunks ah, as UNS. In the case of waves, the energy arrives not as discrete objects but it's, it's continuous. And as we saw in the case of protons and electrons, they behave discreetly. They arrive in discreet chunks which we think off as electrons or photons which are particles of light so discreet. In the case of bullets, we talked about the probability of arrival at x. In the case of waves, we measured the intensity or energy. In the case of electrons or photons, well again we measure the probability of arrival. Which we said is proportional to the intensity. In the case of bullets, when we have both slits open, we saw no interference. In this case of waves, we saw interference. In the case of protons and electrons, we again, have interference and this is the funny thing. So, even though photons and electrons arrive as discreet entities and we think they should have gone through either slit one or slit two, we do get the interference pattern and this is part of the mystery. This is where we have this strange behavior quantum mechanically. In the case of bullets, when both slits are open, n12 is n1 + n2. In the case of waves i12 was not equal to i1 + i2 but what we had was that each one to the height of the wave did add. And the intensity was the square of the height. In the case of photons or electrons, again, we had, i12 is not equal to i1 + i2. The probabilities did not add but then we, we came up with this notion of an amplitude which is just some inverted notion and said, a1,2 is equal to a1 + a2. And that the intensity or probability is just the square of a, and actually put the square inside absolute values because in fact, the amplitude can also be a complex number not just positive and negative.

\end{document}
