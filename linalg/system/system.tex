\listfiles
\documentclass{article}

\usepackage{amsmath}
\usepackage{amssymb}
\usepackage{amsthm}

\usepackage[a4paper,margin=1in]{geometry}

\title{Vectors}
\date{}

\begin{document}
\maketitle

\section{Geometry of Linear Equations}

Given a system of linear equations, what is the solution set?

Let us work in equations of three variables $x, y, z$. We will consider a solution to be a vector in $\mathbb{R}^3$. We assume that each individual linear equation is consistent.

With 0 equations our solution set is $R^3$.

With 1 equation our solution set is a plane.

With 2 equations our solution set is either

\begin{enumerate}\itemsep0pt
\item{a line}
\item{the empty set}
\item{a plane}
\end{enumerate}

The first case is the most common, while the second occurs if the planes are parallel; this corresponds to an inconsistent set of equations. The third occurs if the two equations are "the same".

With 3 equations our solution set is

\begin{enumerate}\itemsep0pt
\item{a point}
\item{a line}
\item{the empty set}
\item{a plane}
\end{enumerate}

notice that there is now a greater variety in how it can be inconsistent. Also, it appears that the only way to have a point as a solution is to have the number of equations greater or equal to the number of unknowns.

\section{Gauss Algorithm}

\theoremstyle{definition}

Now we develop an algebraic way to solve this.

\newtheorem*{GA}{Gauss's Algorithm}
\begin{GA}
If a linear system is changed to another by
one of these operations
\begin{enumerate}\itemsep0pt
\item{an equation is swapped with another ($\rho_i \leftrightarrow \rho_j$)}
\item{an equation has both sides multiplied by a nonzero constant ($n\rho_i$)}
\item{an equation is replaced by the sum of itself and a multiple of another ($n\rho_i + \rho_j$)}
\end{enumerate}
then the two systems have the same set of solutions.
\end{GA}

There is an algorithm (gaussian elimination) that uses these operations to write the equation in reduced row-echelon form

\section{Algebraic proofs}

\subsection*{solution sets are $\mathbb{R}^n$}

\newtheorem*{PH}{Theorem}
\begin{PH}
For any linear system there are vectors $\vec{p}, \vec{\beta_1}...\vec{\beta_k}$ such that the solution set can be described as a $\vec{p}$ and a linear combination of $\beta$.

RREF

A homogeneous system of m linear equations in n un-
knowns always has a non–trivial solution if m < n.


\end{PH}

\subsection*{$m \ge n$ for unique solution}


\section{General = Particular + Homogenous}

Such a split occurs in ordinary differential equations, linear diophantine equations, and many more (vector space!)

\end{document}
