\listfiles
\documentclass{article}

\usepackage{amsmath}
\usepackage{amssymb}
\usepackage{amsthm}

\usepackage[a4paper,margin=1in]{geometry}
\setlength\parindent{0pt}
\setlength\parskip{12pt}

\title{Eigenvectors}
\date{}

\begin{document}
\maketitle

We have a matrix A. Sometimes A behaves very specially on a special kind of vector; it simply lengthens or shortens it. $x$ is parallel to $Ax$. In equations

\begin{align*}
Ax = \lambda x
\end{align*}

$x$ is the eigenvector and $\lambda$ is the eigenvalue.

We can see that $x=0$ always satisfies the equation for any $\lambda$. That's not nice, so we don't consider $0$ to be an eigenvector.

Notice that if $x$ is an eigenvector, so is $kx$ for any $k$. Eigenvectors lie on (at least) a line.

For example, if $P$ is the projection matrix, we can find its eigenvectors by observation: everything that lies on the projection plane will get projected to itself.

\section{Finding eigenvectors}

There is a special case we can consider: $\lambda = 0$, zero eigenvalues. Then the equation becomes

\begin{align*}
Ax = 0
\end{align*}

so $\{x\} = Null(A)$

What about in general? The equation $Ax = \lambda x$ has two unknowns. Let us fix $\lambda$ and rewrite

\begin{align*}
Ax &= \lambda x \\
&= \lambda I x \\
(A-\lambda I) x &= 0
\end{align*}

so $\{x\} = Null(A-\lambda I)$. We see that eigenvectors belonging to the same eigenvalue form a vector space (eigenspace), which explains the $kx$ observation above.

\section{Permutation matrix}
\begin{align*}
A =
\begin{bmatrix} 
0 & 1 \\ 
1 & 0
\end{bmatrix}
\end{align*}

A permutes two elements by swapping. We can see one eigenvector by observation:  $x = (1\ 1), \lambda = 1$. The other one is harder to spot; $x = (1\ -1), \lambda = -1$. Notice that $\lambda_1 \lambda_2 = -1$ is the determinant of A while $\lambda_1 + \lambda_2 = 0$ is the trace.

\section{Finding $\lambda$}
Now that we know how to find $x$ given $\lambda$, how do we find $\lambda$? Remember that $\{x\} = Null(A-\lambda I)$ and $x \neq 0$, we see that $A-\lambda I$ must be singular. So we have to solve the characteristic equation, a polynomial in $\lambda$, $det(A-\lambda I) = 0$. In general it is an n-th order equation for an n-dimensional matrix.

Another way to derive this is to see that if $A \rightarrow A + 3I$ the eigenvalues increase by 3 and the eigenvectors remain unchanged. Since we know how to find eigenvectors of a singular matrix we bring $A$ to a singular matrix in order to find their eigenvectors.

\section{Good Example}

\begin{align*}
A &=
\begin{bmatrix} 
3 & 1 \\ 
1 & 3
\end{bmatrix} \\
A - \lambda I &=
\begin{bmatrix} 
3-\lambda & 1 \\ 
1 & 3-\lambda
\end{bmatrix} \\
det &= \lambda^2 - 6\lambda + 8 \\
&= (\lambda-4)(\lambda-2)
\end{align*}

Notice that $Tr(A) = 6$, $det(A) = 8$ both appear in the characteristic polynomial.

\begin{align*}
A - 2I &=
\begin{bmatrix} 
1 & 1 \\ 
1 & 1
\end{bmatrix} \\
x &= 
\begin{bmatrix} 
1\\ 
-1
\end{bmatrix} 
\end{align*}

\begin{align*}
A - 4I &=
\begin{bmatrix} 
-1 & 1 \\ 
1 & -1
\end{bmatrix} \\
x &= 
\begin{bmatrix} 
1\\ 
1
\end{bmatrix} 
\end{align*}

\end{document}
