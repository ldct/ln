\listfiles
\documentclass{article}

\usepackage{amsmath}
\usepackage{amssymb}

\usepackage[a4paper,margin=1in]{geometry}

\title{Schrodinger}
\date{}

\begin{document}
\maketitle

A mix of experimental and mathematical whacking

\section{Starting assumptions}

Mostly experimental

\begin{align}
p = \hbar k
\end{align}

\section{Schrodinger's equation is linear}

Wave packet

\section{Photons}

For a monochromatic wave in vacuum, with no currents or charges present,
Maxwell’s wave equation,

\begin{align}
\nabla^2 E - \frac{1}{c^2} \partial_t{E} = 0
\end{align}

admits the plane wave solution, 
\begin{align}
\mathbf{E} &= \mathbf{E_0} e^{i(\mathbf{k \cdot r} - \omega t)} \\
           &= \mathbf{E_0} e^{i(\mathbf{p \cdot r} - E t) / \hbar}
\end{align}

where the second equation, expressed in terms of $p$ and $E$, allow us to connect more readily with particles.

\section{TDSE}

Treating a particle as a wave, we write the "phase"

\begin{align}
\psi &= A e^{i(\mathbf{p \cdot r} - E t) / \hbar} \\
\partial_t{\psi} &= i E / \hbar \\
\partial^2_\mathbf{r}{\psi} &= - p^2 / \hbar^2
\end{align}

Using the classical relation that $E = p^2/2m + V$, we get

\begin{align}
i \hbar \partial_t{\psi} = -\frac{\hbar^2}{2m} \partial^2_\mathbf{r}{\psi} + V\psi
\end{align}

\begin{align}
i \hbar \frac{\partial{\psi}}{{\partial t}} = -\frac{\hbar^2}{2m} \frac{\partial^2 \psi}{\partial \mathbf{r}} + V\psi
\end{align}

If we instead use $E^2 = (cp)^2 + m^2c^4$ and do some other stuff we end up with the Klein-Gordon equation

\section{TISE}

If we separate the time dependance to write (can we always do this? it turns out yes) $\psi(x,t) = e^{iEt}\psi(x)$ and V s independent of time then

\begin{align}
-\frac{\hbar^2}{2m} \partial^2_x{\psi} + V\psi = E\psi
\end{align}

for some constants $E$. Various ways to derive this, including from the heuristic, a formal definition to prove that any state can be written as a sum of stationary states, etc.

\section{Particle flux}
???

\end{document}
