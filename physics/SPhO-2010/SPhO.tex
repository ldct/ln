\listfiles
\documentclass{article}

\usepackage[pdftex]{graphicx}
\usepackage{amsmath}
\usepackage{amssymb}

\usepackage[a4paper,margin=1in]{geometry}

\newcommand{\half}{\frac{1}{2}}
\newcommand{\<}{\langle}
\renewcommand{\>}{\rangle}

\setlength\parindent{0pt}
\setlength\parskip{6pt}

\title{}
\date{}

\begin{document}
\maketitle


\section{}

In 1899, Max Planck introduced the units $\hbar, c$ and $G$. 

In terms of these Planck units, write down the dimensions of mass, length and time. These quantities are called the Planck mass, length and time. 

Find the value of $M_{pl}$ in SI units. 

Find the ratio $\frac{E_{grav}}{m_e c^2}$ where $E_{grav}$ is the gravitational energy between two electrons separated by a distance equal to the Compton wavelength of an electron of mass $m_e$.

Consider a particle of mass $M_{pl}$. Find the ratio $\frac{E_{grav}}{M_{pl}c^2}$ where $E_{grav}$ is the gravitational energy between two such particles separated by a distance equal to their own Compton wavelength. Thus, $M_{pl}$ can be interpreted as the mass scale that quantum gravitational effects become important.

\section{}

A block of mass $M$ rests on a fixed plane inclined at an angle $\theta$. A horizontal force of $Mg$ is applied to the block, as shown in Fig 1. The coefficient of static friction between the block an dthe plane is $\mu$

Asuming that the block is at rest, determine $N$ and $F_f$ between block and plane in terms of $M$ and $\mu$, and

determine the range of angles $\theta$ for which the block remains at rest on the plane in terms of $\mu$.

\section{}

A mobile is formed by supporting four metal butterflies of equal mass $m$ from a string of length $L$. The points of support are evenly spaced a distance $l$ apart as shown in Fig 2. The string forms an angle $\theta_1$ with the ceiling at each end point. The center section of string is horizontal.

Find the tension in each section of string in terms of $\theta_1, m$ and $g$.

Find the angle $\theta_2$ in terms of $\theta_1$

Show that the distance $D$ between the end points of the string is $D = \frac{L}{5} (a \cos\theta_1 + b\cos[\tan^{-1}(\half\tan\theta_1)] + 1)$ where $a$ and $b$ are constants to be determined.

\section{}

A 670kg meteorite is composed of aluminum. At a distance far from the Earth, its temperature is -15 degrees C and it moves with a speed of 14.0 kmps relative to the Earth. As it crashes into the planet, the resulting additional internal energy is shared equally between the meteor and the planet. Assuming that all of the material of the metor rises momentarily to the same final temperature, determine this temperature. You may assume that the specific heat of liquid and of gaseous aluminium is....

\section{}

A pion at rest with a mass $m_\pi$ decays to a muon of mass $m_\mu$ and an antineutrino of negligible mass. The reaction is written $\pi^- \rightarrow \mu^- + \bar\nu$. caluculate the kinetic energy of the muon and the energy of the antineutrino in eV. $m_\pi = 273m_e, m_\mu = 207m_e$

\section{}

A small disc of radius $r$ and mass $m$ is attached rigidly to the face of a second larger disk of radius $R$ and mass $M$ as shown in Fig 3. The centre of the small disk is located at the edge of the large disk. The large disk is mounted at its center on a frictionless axis. The assembly is rotated through a small angle $\theta$ from its equillibrium position and releated.

Show that the speed of the center of the small disk as it passes through the equilibrium position is $v = \alpha[\frac{Rg(1 - \cos\theta)}{(M/m) + (r/R)^2 + \beta}]^\half$ where $\alpha$ and $\beta$ are constants to be determined.

Determine the period of the motion in terms of $M, m, R$ and $r$.

\section{}

An electric motor turns a flywheel through a drive belt that joins a pulley on the motor and a pulley that is rigidly attached to the flywheel, as shown in Fig 4. The flywheel is a solid disk with a mass of 80kg and a diameter of 1.25m. It turns on a frictionless axle. Its pulley has a much smaller mass and a radius of 0.230m. If the tension in the upper (taut) segment of the belt is 135N and the flywheel has a colckwise angular acceleration of 1.67 rad s-2, find the tension in the lower (slack) segment of the belt.

\section{}

A plano-concave lens having index of refraction 1.50 is placed on a flat glass plate, as shown in Fig 5. Its curved surface, with radius of curvature 8.00m, is on the bottom. The lens is illuminated from above with yellow sodium light of wavelength 589nm, and a serien of concentric bright and dark rings is observed by reflection. The interference pattern has a dark spot at the center, surrounded by 50 dark rings, of which the largest is at the outer edge of the lens.

i) What is the thickness of the air layer at the center of the interference pattern?
ii) Calculate the radius of the outermost dark ring.
iii) Find the focal length of the lens.

\section{}

a) A toroid has a major radius $R$ and a minor radius $r$ and it is tightly wound with $N$ turns of wire, as shown in Fig 6. If $R >> r$, the magnetic field in the region encnlosed by the wire of the torus, of cross-sectional area $A=\pi r^2$, is essentially the same as the magnetic feild of a solenoid that has been bent into a large circle of radius $R$.

Show that the self-inductance of such a toroid is approximately $L \approx \kappa\mu_0\frac{N^\alpha A}{R}$ where $\kappa$ and $\alpha$ are constants.

b) The toroid in Fig 6 with $N$ turns of wire is now replaced by one with a rectangular cross section. Its inner and outer radii are $a$ and $b$, respectively. The cross-section is a rectangle of length $b-a$ and breadth $h$.

i) Show that the inductance of the toroid is $L = \kappa'\mu_0\frac{N^\beta h}{R} \ln\frac{b}{a}$ where $\kappa' and \beta$ are constants.

ii) Compute the self-inductance of a 500-turn toroid for which $a=10.0cm, b=12.0cm$, and $h=1.00cm$. In part (a), an approximate expression for the inductances of a toroid with $R >> r$ was derived. If the calculations in part (b)(ii) were done using this approximate expression for self-inductance, what is the percentage error in the result?

\section{}

An empty box of total mass $M$ with perfectly reflecting walls is at rest in the lab frame. Then electromagnetic standing waves are introduced along the $x$ direction, consisting of $N$ photons, each of frequency $\nu$ as shown in Fig 8. Determine the rest mass of the system (box + photons) when the photons are present.


\end{document}
