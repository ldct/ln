\listfiles
\documentclass{article}

\usepackage{amsmath}
\usepackage{amssymb}

\usepackage[a4paper,margin=1in]{geometry}
\newcommand{\sgn}{\operatorname{sgn}}
\newcommand{\mb}{\mathbf}


\title{Basics}
\date{}

\begin{document}
\maketitle

write Newton's law for the $i$th particle.

\begin{align}
\mb{F}_i &= \dot{\mb{p}}_i
\end{align}

expanding out $F_i$ as $\sum{\mb{F}_{ji}} + \mb{F}^e_i$, where $\mb{F}_{ji}$ is the constraint force of particle $j$ on $i$,

\begin{align}
\sum_j{\mb{F}_{ji}} + \mb{F}^e_i &= \dot{\mb{p}}_i
\end{align}

sum this over i,

\begin{align}
\sum_{i,j}{\mb{F}_{ji}} + \sum_i\mb{F}^e_i &= \sum_i\dot{\mb{p}}_i
\end{align}

due to the law of action and reaction, $\mb{F_{ji} = -F{ij}}$, the sum over pairs of reaction forces cancel out, so the first term is 0. The second term we write as $\mb{F^e}$, or total external force. To interpret the third term we define the center-of-mass vector $\mb{R}$ as the mass-weighted sum of the $r_i$,

\begin{align}
\mb{R} &= \frac{\sum{m_i \mb{r}_i}}{\sum{m_i}}
\end{align}

to obtain

\begin{align}
\mb{F}^e = M\mb{\ddot{R}}
\end{align}


\end{document}
