\listfiles
\documentclass{article}

\usepackage{amsmath}
\usepackage{amssymb}

\usepackage[a4paper,margin=1in]{geometry}
 \newcommand{\sgn}{\operatorname{sgn}}

\title{Lagrangian}
\date{}

\begin{document}
\maketitle

source: http://physics.stackexchange.com/questions/885/why-does-calculus-of-variations-work

\section{Independence}

Why is the Lagrangian a function of two variables? Put differently, you can choose position and velocity independently as initial conditions, that's why the Lagrangian function treats them as independent; but the calculus of variation does not vary them independently, a variation in position induces a fitting variation in velocity.

\begin{align}
S &= \int L(q,\dot{q},t) dt \\
\delta S &= \int \frac{\partial L}{\partial q} \delta q + \frac{\partial L}{\partial \dot{q}} dt = 0
\end{align}

the second step seems suspect; why are $q$ and $\dot{q}$ varied independently?

now 

\begin{align}
\delta \dot{q} = \frac{d}{dt}\delta q
\end{align}

the independence of $q$ and $\dot{q}$ is removed by this identity.

\begin{align}
S &= \int L(q,\dot{q},t) dt \\
\delta S &= \int \frac{\partial L}{\partial q} \delta q + \frac{\partial L}{\partial \dot{q}} dt = 0
\end{align}

\begin{align} \delta S = \lbrack {\partial L \over \partial \dot q}\delta q\rbrack_{t_1}^{t_2} + \int_{t_1}^{t_2} ({\partial L \over \partial q} - {d \over dt}{\partial L \over \partial \dot q})\delta q dt = 0
\end{align}

and the bracketed expression is zero because the endpoints are held fixed. And then we can pull out the Euler-Lagrange equation:

\begin{align} {\partial L \over \partial q} - {d \over dt}{\partial L \over \partial \dot q} = 0. \end{align}

Now it makes more sense to me. You start by treating the variables as independent but then remove the independence by imposing a condition *during* the derivation.



\end{document}
