\listfiles
\documentclass{article}

\usepackage{amsmath}
\usepackage{amssymb}

\usepackage[a4paper,margin=1in]{geometry}

\title{Laplacian}
\date{}

\begin{document}
\maketitle

\section{$\nabla$}

\newcommand{\dels}{$\nabla^2$}
\newcommand{\delsm}{\nabla^2}

We assume you are familiar with the humble $\nabla$ (del) vector and its physcal interpretation, including in the operator form $\nabla$ (scalar $\rightarrow$ vector), $\nabla\cdot$ (vector $\rightarrow$ scalar) and $\nabla\times$ (vector $\rightarrow$ vector).

There are 9 (3 $\cdot$ 3) combinations of two del operators. Three of them have the correct type signature:

\begin{enumerate}
\item{$\nabla\times\nabla\times$}
\item{$\nabla\times\nabla$}
\item{$\nabla\cdot\nabla\times$}
\item{$\nabla\cdot\nabla$}
\item{$\nabla\nabla\cdot$}
\end{enumerate}

Items 2 and 3 are always zero.

\section{\dels}

The \dels operator, otherwise as the laplacian operator, is very important operator in physics. In cartesian coordinates it is

\begin{eqnarray*}
\delsm = \frac{\partial^2}{\partial x^2} + \frac{\partial^2}{\partial y^2} + \frac{\partial^2}{\partial z^2}
\end{eqnarray*}

Note that this value is necessarily a scalar, ie, independent of coordinates chosen. So what is its physical interpretation?

\section{Concavity}

In single-variable calculus the second derivative represents concavity, or the rate of change of the rate of change.

A slightly more useful way to think about this is in terms of average values; the concavity of a function $f$ at a point $x_0$ measures much the average value of $f(x)$ about $x_0$ exceeds $f(x_0)$. [pics]

This argument generalizes to higher dimensions too.

\section{Poisonn's equation}

\begin{eqnarray*}
\delsm = 0
\end{eqnarray*}

\end{document}
