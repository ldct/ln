\listfiles
\documentclass{article}

\usepackage[pdftex]{graphicx}
\usepackage{amsmath}
\usepackage{amssymb}

\usepackage[a4paper,margin=1in]{geometry}

\newcommand{\half}{\frac{1}{2}}

\title{The Bohr's Molecule}
\date{}

\begin{document}
\maketitle

\section{$\mathrm{H_2}$ molecule in equillibrium}

\subsection{$r/R$ and $\rho/R$}

Let $\theta$ be the angle subtended by the proton-electron line from the proton-proton line, i.e. the angle between the lines with length $R$ and $r$ respectively. Then, balancing the repulsive and attractive forces,

\begin{align*}
\frac{ke^2}{R^2} &= 2 \frac{ke^2}{r^2} \cos\theta
\end{align*}

noting that $\cos\theta = \frac{r}{2R}$,

\begin{align*}
\frac{1}{R^2} &= \frac{r/R}{r^2} \\
\frac{1}{R^2} &= \frac{r}{R} \\
R &= r
\end{align*}

hence the two protons and one electron form the vertices of an equilateral triangle. This makes sense as now the forces are of equal magnitude and all are $120^\circ$ from each other. Hence $\theta = \frac{\pi}{3}$ and

\begin{align*}
\frac{r}{R} = 1
\end{align*}

\begin{align*}
\frac{\rho}{R} &= \sin\frac{\pi}{3} \\
&= \frac{\sqrt 3}{2}
\end{align*}

\subsection{$E_p$}

Let $E_{pp}$ be the electric potential energy between a pair of protons in the configuration, $E_{pe}$ be the electric potential energy between a proton and an electron, and $E_{ee}$ be that between two electrons. Then

\begin{align*}
E_{pp} &= \frac{ke^2}{R} \\
E_{pe} &= -\frac{ke^2}{r} \\
&= -\frac{ke^2}{R} \\
E_{ee} &= \frac{ke^2}{2\rho} \\
&= \frac{ke^2}{2R} \frac{2}{\sqrt 3} \\
&= \frac{ke^2}{R} \frac{\sqrt 3}{3}
\end{align*}

since there is one pair of protons, one of electrons and 4 proton-electron pairs,

\begin{align*}
E_p &= E_{pp} + E_{ee} + 4 E_{pe} \\
&= \frac{ke^2}{R} \left(\frac{\sqrt 3}{3} - 3\right)
\end{align*}

\subsection{$E_p/E_k$}

Let $F_c$ be the centripetal force pulling the electron down. Then

\begin{align*}
F_c &= 2 \cos \frac{\pi}{6} \frac{ke^2}{\rho^2} \\
&= m_e \frac{v^2}{\rho} \\
\end{align*}

Hence

\begin{align*}
E_k &= \frac{1}{2} m_e v^2 \\
&= \cos \frac{\pi}{6} \frac{ke^2}{\rho} \\
&= \frac{ke^2}{R}
\end{align*}

So

\begin{align*}
\frac{E_k}{E_p} &= \frac{\frac{ke^2}{R}}{\frac{ke^2}{R} \left(\frac{\sqrt 3}{3} - 3\right)} \\
&= -\frac{\sqrt 3 + 3}{26}
\end{align*}

\subsection{$R_0$}

Now

\begin{align*}
\frac{1}{2} m_e v^2 = \frac{ke^2}{R}
\end{align*}

and so

\begin{align*}
v = e\sqrt{\frac{2k}{m_e R}}
\end{align*}

the momentum $p$ is

\begin{align*}
p &= m_e v \\
&= e\sqrt{\frac{2km_e}{R}}
\end{align*}

since the momentum of the electron about the center of the molecule is perpenducal to the radial vector to the electron, $L = rp$ and

\begin{align*}
L &= R p \\
&= e\sqrt{2km_eR} \\
&= n\hbar
\end{align*}

$R$ will be minimized when $n$ is the smallest, which occurs when $n=1$

\begin{align*}
e\sqrt{2km_eR_0} &= \hbar \\
R_0 &= \frac{\hbar^2}{2km_e e^2}
\end{align*}

\subsection{$E_b$}

The total energy of the molecule is

\begin{align*}
E &= E_k + E_p \\
&= \frac{ke^2}{R} \left(1 + \frac{\sqrt 3}{3} - 3\right) \\
&= \frac{ke^2}{R} \left(\frac{\sqrt 3}{3} - 2\right)
\end{align*}

wheares the energy of the hydrogen atom is $E_I$. Hence the binding energy, which is the difference between the energy of a $\mathrm{H_2}$ molecule and two $H$ atoms, is

\begin{align*}
E_b &= E - 2 E_I \\
&= \frac{ke^2}{R} \left(\frac{\sqrt 3}{3} - 2\right) - 2 E_I
\end{align*}

\section{Vibrating $\mathrm{H_2}$ molecule}

\subsection{Morse potential}
\subsection{$D$ and $E_b$}

\begin{align*}
\mathrm{When}\ R = R_0&, E = -D \\
\mathrm{As}\ R \rightarrow \infty&, E \rightarrow 0
\end{align*}

Hence the bonding energy $E_b = D$.

\subsection{Linear frequency of vibration}

\end{document}
