\listfiles
\documentclass{article}

\usepackage{amsmath}
\usepackage{amssymb}
\usepackage{mathtools}
\usepackage{listings}
\usepackage{hyperref}

\DeclarePairedDelimiter\floor{\lfloor}{\rfloor}
\DeclarePairedDelimiter\ceil{\lceil}{\rceil}
\DeclareMathOperator{\cl}{cl}
\DeclareMathOperator{\E}{E}
\def\Z{\mathbb{Z}}
\def\N{\mathbb{N}}
\def\R{\mathbb{R}}
\def\Q{\mathbb{Q}}
\def\K{\mathbb{K}}
\def\T{\mathbb{T}}
\def\B{\mathcal{B}}
\def\XX{\mathfrak{X}}
\def\YY{\mathfrak{Y}}
\def\AA{\mathfrak{A}}
\def\ZZ{\mathfrak{Z}}
\def\BB{\mathcal{B}}
\def\UU{\mathcal{U}}
\def\MM{\mathcal{M}}
\def\M{\mathfrak{M}}
\def\l{\lambda}
\def\L{\Lambda}
\def\<{\langle}
\def\>{\rangle}
\def\f12{\frac{1}{2}}
\def\inv{{-1}}
\def\im{\textrm{im}}
\def\Stab{\textrm{Stab}}

\usepackage[a4paper,margin=1in]{geometry}

\setlength{\parindent}{0cm}
\setlength{\parskip}{1em}

\title{HW 4}
\date{}

\begin{document}
\maketitle

\section*{1}

\subsection*{1.7.18}

Reflexive: for all $a \in A$ we have $a \sim a$ since $a = e a$.

Symmetric: let $a, b \in A$ such that $a \sim b$, that is, $a = h b$. Then $b = h^\inv a$. Since $h^\inv \in H$ this means $b \sim a$.

Transitive: let $a, b, c \in A$ such that $a \sim b, b \sim c$. This means $a = h_1 b, b = h_2 c$ for some $h_1, h_2 \in H$. Then $a = h_1 \cdot (h_2 \cdot c) = (h_1 h_2) \cdot c$ hence $a \sim c$ because $h_1 h_2 \in H$.

\subsection*{1.7.19}

Let $\phi$ be the map.

Injective: suppose $\phi(h_1) = \phi(h_2)$, that is $h_1 x = h_2 x$. By multiplying by $x^\inv$ on the right, we have $h_1 = h_2$.

Surjective: let $y \in O$ be some element in the codomain of $\phi$. This means $x \sim y$, that is there exists some $h$ with $x = hy$. Then $\phi(h^\inv) = h h^\inv y = y$. Here $\phi(h^\inv)$ is well-defined because $h^\inv \in H$.

Let $O_g$ be the orbit of $g$. The bijection given by $\phi$ tells us that $|O_g| = |H|$. Since the orbits partition $G$ we can write $G = O_{g_1} \sqcup O_{g_2} \ldots \sqcup O_{g_k}$ for some subset $\{g_1, g_2 \ldots g_k\} \subseteq G$ which means $|G| = \sum_k |O_{g_k}| = k |H|$.

\section*{2}

\subsection*{a}

Let $D_{14}$ act faithfully on $A$ where $n = |A| < 7$. The group action is equivalent to a group homomorphism $D_{14} \to S_A$. Since the action is faithful, this is an injective homomorphism (since distinct elements of $D_{14}$ are mapped to distinct permutations). By Cayley's theorem we have $S_A$ is a subgroup of $S_6$; hence there is an injective homomorphism $D_{14} \to S_6$; the range of this homomorphism is a subgroup of $S_6$ isomorphic to $D_{14}$. By Lagrange's theorem the order of this subgroup divides $|S_6| = 6!$, that is $14 | 6!$, a contradiction.

\subsection*{b}

We wish to construct an isomorphic copy of $D_{12}$ in $S_5$, say with generating permutations $r, s$ satisfying the usual relations. We have $ord(r) = 6$ hence $r$ must decompose into a 3-cycle multiplied by a 2-cycle; WLOG $r = (1,2,3)(4, 5)$. If we take $s = (1, 2)$ we have $rsrs = (1,2,3)(4, 5)(1, 2)(1,2,3)(4, 5)(1, 2) = (1,2,3)(1,2)(1,2,3)(1,2) = (1,2,3)(2,1,3) = e$ and $s^2 = e$.

Concretely, the action can be defined as follows: $r^i s^j \cdot x = (1, 2, 3)^i(1, 2)^j(4, 5)^ix$ for $i \in [0, 6), j \in [0, 1)$.

\subsection*{c}

By an argument similar to 2a, for $n > 2$ if $D_{2n}$ acts faithfully on a set with $k$ elements then there is an injective homomorphism $D_{2n} \to S_k$, in particular $n | k!$. For $(n, k) = (7, 6)$ this is a contradiction; the smallest factorial which is a multiple of $2 \cdot 7$ is $7!$. In general for $(n, k) = (p, p-1)$ for $p$ prime this leads to a contradiction. However for $(n, k) = (6, 5)$ there is no problem, since $6 | 5!$.

\section*{3}

Lemma: let $H \le G, h \in G$ and let $\phi: H \to G$ be conjugation by $h$. Claim: $\phi$ is an injective group homomorphism. Proof: for $h_1, h_2 \in H, \phi(h
_1)\phi(h_2) = hh_1h^\inv hh_2h^\inv = hh_1h_2h^\inv = \phi(h_1 h_2)$. Furthermore $\phi(h_1) = \phi(h_2) \iff hh_1h^\inv = hh_2h^\inv \iff h_1 = h_2$.

\subsection*{a}

Let $G_x, G_y$ be two stabalizers.

For every $h \in G, a, b \in A$ such that $h \cdot a = b$ let $\phi: G_a \to G$ be conjugation by $h$, which is a group homomorphism.

Claim: $\im \phi = G_b$. Proof: $g \in G_a \iff g \cdot a = a \iff h g h^\inv \cdot b = b \iff hgh^\inv \in G_b$ where the second biimplication follows because $hgh^\inv \cdot b = hg \cdot a = h \cdot a = b$.

Diagramatically, the stable action on $b$ corresponds to travelling to $a$, performing a stable action, and then traveling back to $b$.

Hence $G_a$ and $G_b$ are isomorphic as long as $a$ and $b$ are in the same orbit; hence if $G$ acts transitively on $A$, all the stabalizers are isomorphic.

\subsection*{b}

We can use $D_8$ acting on the seven binary squares (slide 5 of \url{https://www.math.clemson.edu/~macaule/classes/s24_math4120/slides/math4120_slides_chapter05_h.pdf})

$\Stab(0,0,0,0) = D_8$ but $r \not\in \Stab(0,1,1,0)$.

\section*{4}

\section*{a}

WLOG we can prove this for transitive actions, since the stabalizers $G_o$ of $o \in O$ in the action of $G$ on $O$ are exactly the same as the stabalizers $G'_o$ of $o \in S$ in the action of $G$ on $S$.

Fix $s$ and consider the set-function $\phi : G \to O$ defined by $\phi(g) = g \cdot s$. This function is surjective since $O$ is transitive. For each $o \in O$ consider the preimage $\phi^\inv(o) = \{g \in G | g \cdot s = o \}$. There exists $g_{o \to s} \in G$ such that $g_{o \to s} \cdot o = s$. Define $g_{s \to o} = g_{o \to s}^\inv$; it is easy to check that $g_{s \to o} \cdot s = o$.

The set $g_{o \to s} \phi^\inv(o) = \{g_{o \to s} x | x \in \phi^\inv(o)\}$ has the same cardinality as $\phi^\inv(o)$ (since left-multiplication in $G$ is invertible) and each element satisfies $g_{o \to s}x \cdot s = g_{o \to s} \cdot o = s$, hence $g_{o \to s}\phi^\inv(o) \subseteq G_s$. Also every $g \in G_s$ can be written as $g_{o \to s} g_{s \to o} g$ where $g_{s \to o}g \in \phi^\inv(o)$.

Hence $g_{o \to s}\phi^\inv(o) = G_s$ and $\phi$ is a surjection onto $O$ where every preimage has size $G_s$, hence $|G| = |G_s| |O|$.

\subsection*{b}

Let $\phi: G \times G \to G$ be an action of $G$ on itself by conjugation, that is $g \cdot a = gag^\inv$. This is a group action because for $h, g, a \in G$ we have $h \cdot (g \cdot a) = h \cdot g a g^\inv = hg ag^\inv h^\inv = hg \cdot a$. A conjugacy class in an orbit under this action, hence by (a) it divides $|G|$.

\subsection*{c}

Suppose $|G| = p^n$ for some prime $p$ and $Z(G) = \{e\}$. For all $g \in G$ let $[g]$ be the set of conjugates of $g$. We have $g \in Z(G) \iff \forall a \in G, a g a^\inv = g \iff [g] = \{g\}$; hence $|[g]| = 1 \iff g = e$. By 4b, $|[g]|$ divides $p^n$. Hence we can write $G$ as a disjoint union of its conjugates $G = [e] \sqcup [g_1] \sqcup [g_2] \ldots [g_k]$ where for each $i, |g_i|$ divides $p^n$ but is not equal to one, hence it is a multiple of $p$. Hence consider the equation $|G| = |[e]| + |[g_1]| + \ldots$ modulo $p$; this becomes $0 = 1 + 0 + 0 + \ldots$, a contradiction.

% 4.
% (a) Let G be a finite group acting on a set S, and take some s\in S. If \mathcal{O} is the orbit of some element s\in S, prove that |\mathcal{O}|=|G|/|G_s|, where G_s denotes the stabilizer. (This is called the orbit-stabilizer theorem.)

% (b) Given a finite group G and an element a\in G, the conjugacy class of a is the set of all elements of G which are conjugate to a. Prove that the size of a conjugacy class always divides the size of the group.

% (c) Deduce from this that, if |G| is a prime power, then the center Z(G) is nontrivial. [This result appears later in Dummitt and Foote, but try to prove it yourself without looking ahead! Ask me if you need a hint.]

\end{document}
