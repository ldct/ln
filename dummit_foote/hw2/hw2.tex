\listfiles
\documentclass{article}

\usepackage{amsmath}
\usepackage{amssymb}
\usepackage{mathtools}
\usepackage{listings}

\DeclarePairedDelimiter\floor{\lfloor}{\rfloor}
\DeclarePairedDelimiter\ceil{\lceil}{\rceil}
\DeclareMathOperator{\cl}{cl}
\DeclareMathOperator{\E}{E}
\def\Z{\mathbb{Z}}
\def\N{\mathbb{N}}
\def\R{\mathbb{R}}
\def\Q{\mathbb{Q}}
\def\K{\mathbb{K}}
\def\T{\mathbb{T}}
\def\B{\mathcal{B}}
\def\XX{\mathfrak{X}}
\def\YY{\mathfrak{Y}}
\def\AA{\mathfrak{A}}
\def\ZZ{\mathfrak{Z}}
\def\BB{\mathcal{B}}
\def\UU{\mathcal{U}}
\def\MM{\mathcal{M}}
\def\M{\mathfrak{M}}
\def\l{\lambda}
\def\L{\Lambda}
\def\<{\langle}
\def\>{\rangle}
\def\f12{\frac{1}{2}}

\usepackage[a4paper,margin=1in]{geometry}

\setlength{\parindent}{0cm}
\setlength{\parskip}{1em}

\title{HW 2}
\date{}

\begin{document}
\maketitle

\section*{1.2}

\section*{5.}

Let $x = s^kr^i$ be an element which commutes with every element of $D_{2n}$ with $k \le 1, k < n$. We have $s^k s = s s^k = s^{1-k}$ since $s$ has order $2$. Since $x$ commutes with $s$,

% align


\begin{align*}
s s^kr^i &= s^kr^i s \\
s^{1-k} r^i &= s^k s r^{-i} \\
&= s^{k+1} r^{-i}
\end{align*}

By equating exponents of $s$ (which we can do because the representation is unique), $1-k = k+1$ hence $k=1$. By equating exponents of $r$, we have $i = -i \pmod n$ hence $2i = 0 \pmod n$ hence $i = 0 \pmod n$ because $n$ is odd. Hence $x$ is the identity.

\section*{7.}

$s^2 = a^2 = 1$

$r^n = (s^2 r)^n = (ab)^n = 1$

$rsr = s(sr)(sr) = ab^2 = a = s$. Hence $rs = sr^{-1}$

Conversely,

$a^2 = s^2 = 1$

$b^2 = (sr)^2 = srsr = ss = 1$

$(ab)^n = (ssr)^n = r^n = 1$

\section*{1.3}

\section*{10.}

This problem needs the convention $a_0 = a_m$ to be true.

Consider the list $[a_1, a_2, \ldots a_m, a_{m+1} = a_1, \ldots]$ formed by repeating copies of $[a_1, a_2, \ldots a_m]$; label the elements $b_1, b_2 \ldots$. For all natural numbers $j$ we have $\sigma(b_j) = b_{j+1}$. Hence $\sigma^i(b_k) = b_{i+k}$. In particular $\sigma^i(a_k) = b_{i+k}$. Then $b_{i+k} = a_{j'}$ where $j'$ is $i+k$ replaced by its least residue mod $m$ where $k+i > m$ by construction of the $b$'s.

We have $\sigma^m(a_k) = b_{m+k} = a_k$ hence $\sigma^m$ is the identity permutation. If $p = ord(\sigma) < m$, then $a_1 = \sigma^{p}(a_1) = a_{p+1}$ where $p+1$ cannot be reduced further, contradicting the fact that the $a$'s are distinct.

\section*{11.}

By relabelling, this is equivalent to proving it for the $m$-cycle $\sigma = (0, 1, 2, \ldots m-1)$. Consider the sequence $[0, i, 2i, \ldots (m-1)i]$ where each element is reduced mod $m$. Each element of this sequence is generated by applying $\sigma^i$ to the previous.

First we show that if $m$ and $i$ are coprime, the sequence consists of distinct elements; then the sequence is the cycle decomposition of $\sigma^i$. Supposing otherwise, let $ai = bi$ be two distinct elements of the sequence with $0 \le a < b < m$. Then $ai = bi \pmod m$ hence $a = b \pmod m$ which is a contradiction.

Conversely, if $gcd(m, i) = g > 1$, then let $k = \frac{m}{g} < m$. We have $ki = \frac{mi}{g} = \frac{lcm(m, i)gcd(m, i)}{g} = lcm(m, i)$. In particular $m | ki$ hence $ki = 0 \pmod m$. Then the $k$-th element of the sequence is $0$ and the cycle decomposition of $\sigma^i$ contains a $k$-cycle.

\section*{15.}

We first prove exercise 24: if $a, b$ commute then $(ab)^n = a^nb^n$. First we consider $n \ge 0$.

Lemma: $b^n$ and $a$ commute. We prove this by induction on $n$. The base case $n=0$ is trivial. Suppose the result holds for $n$. Then $b^{n+1}a = bb^na = bab^n = abb^n = a b^{n+1}$.

Next we do induction on $n$. The base case $n=0$ is trivial. Suppose the result holds for $n$. Then $(ab)^{n+1} = (ab)^n(ab) = a^nb^na b = a^n a b^n b = a^{n+1}b^{n+1}$.

For negative $n$, let $p = -n$, we need to prove $(ab)^{-p} = a^{-p}b^{-p}$, equivalently $e = a^{-p}b^{-p} (ab)^p$, equivalently $(ab)^p = a^pb^p$ which is true by the positive $n$ case.

Main theorem: Let $x \in S_n$ be a permutation with (disjoint) cycle decomposition $x = c_1c_2\ldots c_k$. Let $n \ge 0$. Since disjoint cycles commute, $x^n = c_1^n c_2^n \ldots c_k^n$ and these cycles are still disjoint. For $n = ord(x)$, each $c_i^n$ must be the identity (if $c_j^n$ is not the identity, let $q$ be some element that $c_j^n$, does not fix, then $x^n$ will not fix $q$ either), hence $ord(c_i) | ord(x)$. For $n = lcm(ord(c_1), \ldots ord(c_k))$, we have each $c_1^n = 1$ hence $x^n = 1$.

\section*{17.}

Each such permutation can be written as $(a, b)(c, d)$ where $a, b, c, d$ are distinct and $a < c$. This is in a 3-1 bijection with unordered 4-tuples of $[1, n]$ since the tuple $p < q < r < s$ corresponds to the permutations $(p, q)(r, s)$, $(p, r)(q, s)$ and $(p, s)(q, r)$. Hence there are $3*\binom{n}{4}$ such permutations.

\section*{6a.}

It suffices to show that each cycle $(a_1, a_2, \ldots a_k)$ is generated by 2-cycles. This is true because $(a_1, a_2, \ldots a_k) = (a_1, a_2)(a_2, a_3)\ldots(a_{k-1}, a_k)$, which we can see by having it operate on an arbitrary $a_j$ from right to left. The first 2-cycle that does not fix its argument is $(a_j, a_{j+1})$ which sends it to $a_{j+1}$. Now the next 2-cycle is $(a_{j-1}, a_j)$ which fixes $a_{j+1}$; every other 2-cycle also fixes it since the indices are decreasing.

\section*{6b.}

The identity has 0 inversions and is even.

A transposition has an even number of inversions. Proof: let the transposition be $(a, b)$ with $a < b$. If $[i, j]$ is a pair with $i < j$ and $i < a$, by considering cases on how $j$ compares with $a$ and $b$, $\sigma(i) < \sigma(j)$. Hence WLOG we can prove that for $S_n$ the permutation $(1, n)$ is even. 

Every pair of indices is one of the following 3 disjoint cases:

If $1 < i < j < n$ then $[i, j]$ is not an inversion.

Let $1 < i < n$. Then $[1, i]$ and $[i, n]$ are both inversions, and are both distinct.

Lastly, $[1, n]$ is an inversion.

Since the inversions in case 2 come in pairs, there is an odd number of inversions.

\end{document}
