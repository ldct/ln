\listfiles
\documentclass{article}

\usepackage{amsmath}
\usepackage{amssymb}
\usepackage{mathtools}
\usepackage{listings}

\DeclarePairedDelimiter\floor{\lfloor}{\rfloor}
\DeclarePairedDelimiter\ceil{\lceil}{\rceil}
\DeclareMathOperator{\cl}{cl}
\DeclareMathOperator{\E}{E}
\def\Z{\mathbb{Z}}
\def\N{\mathbb{N}}
\def\R{\mathbb{R}}
\def\Q{\mathbb{Q}}
\def\K{\mathbb{K}}
\def\T{\mathbb{T}}
\def\B{\mathcal{B}}
\def\XX{\mathfrak{X}}
\def\YY{\mathfrak{Y}}
\def\AA{\mathfrak{A}}
\def\ZZ{\mathfrak{Z}}
\def\BB{\mathcal{B}}
\def\UU{\mathcal{U}}
\def\MM{\mathcal{M}}
\def\M{\mathfrak{M}}
\def\l{\lambda}
\def\L{\Lambda}
\def\<{\langle}
\def\>{\rangle}
\def\f12{\frac{1}{2}}

\usepackage[a4paper,margin=1in]{geometry}

\setlength{\parindent}{0cm}
\setlength{\parskip}{1em}

\title{HW 2}
\date{}

\begin{document}
\maketitle

\section*{1.2}

\section*{5.}


Let $x = s^kr^i$ be an element which commutes with every element of $D_{2n}$ with $k \le 1, k < n$. We have $s^k s = s s^k = s^{1-k}$ since $s$ has order $2$. Since $x$ commutes with $s$,

% align


\begin{align*}
s s^kr^i &= s^kr^i s \\
s^{1-k} r^i &= s^k s r^{-i} \\
&= s^{k+1} r^{-i}
\end{align*}

By equating exponents of $s$ (which we can do because the representation is unique), $1-k = k+1$ hence $k=1$. By equating exponents of $r$, we have $i = -i \pmod n$ hence $2i = 0 \pmod n$ hence $i = 0 \pmod n$ because $n$ is odd. Hence $x$ is the identity.

\section*{7.}

$s^2 = a^2 = 1$

$r^n = (s^2 r)^n = (ab)^n = 1$

$rsr = s(sr)(sr) = ab^2 = a = s$. Hence $rs = sr^{-1}$

Conversely,

$a^2 = s^2 = 1$

$b^2 = (sr)^2 = srsr = ss = 1$

$(ab)^n = (ssr)^n = r^n = 1$

\section*{1.3}

\section*{11.}



\section*{15.}

\section*{17.}


\end{document}
