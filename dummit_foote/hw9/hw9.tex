\listfiles
\documentclass{article}

\usepackage{amsmath}
\usepackage{amssymb}
\usepackage{mathtools}
\usepackage{listings}
\usepackage{hyperref}
\usepackage[all,pdf]{xy}
\usepackage{lmodern,amssymb}

\DeclarePairedDelimiter\floor{\lfloor}{\rfloor}
\DeclarePairedDelimiter\ceil{\lceil}{\rceil}
\DeclareMathOperator{\cl}{cl}
\DeclareMathOperator{\E}{E}
\def\Z{\mathbb{Z}}
\def\N{\mathbb{N}}
\def\R{\mathbb{R}}
\def\Q{\mathbb{Q}}
\def\K{\mathbb{K}}
\def\T{\mathbb{T}}
\def\B{\mathcal{B}}
\def\XX{\mathfrak{X}}
\def\YY{\mathfrak{Y}}
\def\AA{\mathfrak{A}}
\def\ZZ{\mathfrak{Z}}
\def\BB{\mathcal{B}}
\def\UU{\mathcal{U}}
\def\MM{\mathcal{M}}
\def\M{\mathfrak{M}}
\def\l{\lambda}
\def\L{\Lambda}
\def\<{\langle}
\def\>{\rangle}
\def\f12{\frac{1}{2}}
\def\inv{{-1}}
\def\im{\textrm{im}}
\def\Stab{\textrm{Stab}}
\def\Dic{\textrm{Dic}}

\usepackage[a4paper,margin=1in]{geometry}

\setlength{\parindent}{0cm}
\setlength{\parskip}{1em}

\title{HW 9}
\date{}

\begin{document}
\maketitle

\section*{4.1.3}

\section*{4.2.9}

Let $p$ be a prime, $G$ have prime power order, $H \le G$ with $[G:H] = p$. By corollary 4.2.5, since $p$ is the smallest prime dividing $|G|, H$ is normal.

Let $G$ be a group of order $p^2$. By Cauchy's theorem, there is an element of order $p$, say $r$. Then the index of $\<r\>$ is $p^2 / p = p$. By the above theorem, $\<r\>$ is normal in $G$.

\section*{4.2.10}

Let $G$ be a non-abelian group of order 6. By Cauchy's theorem, $G$ has an element of order 2, say $s$. Let $H = \<s\>$ and consider the action of $G$ on the left cosets of $H$, which we can label $H, H', H''$, with associated homomorphism $\phi_H \le S_{\{H, H', H''\}} \cong S_3$. The stababilzer of $H$ is $G_H = H$, and all the stabalizers are conjugate to $G_H$. 

By theorem 4.2.3.3, $\ker \pi_H$ is some subgroup of $H$, so it is etiher either $\{e\}$ or $H$. If it is $\{e\}$ then $\phi_H$ is injective and there is an injective homomorphism $G \to S_3$; since both sides have size 6 we have $G \cong S_3$. 

Otherwise, since $\ker \phi_H = H$ is the intersection of every stabalizer, every stabalizer must be $H$, so $H$ is a normal subgroup. It remains to be shown that in this case, $G \cong C_6$.

Proof 1: Let $g \in G$ be arbitrary. Then $gH = Hg$, which means $\{g, gs\} = \{g, sg\}$, which means $gs = sg$, hence $H$ commutes with every element of $G$. By Cauchy's theorem, $G$ has an element of order 3, say $r$, with $rs = sr$. The subgroup $\<r, s\>$ has order 6 (since it contains subgroups of order 2 and 3), with presentation $\<r, s | rs = sr, r^3 = s^2 = e, \ldots \>$ where $\ldots$ indicates some additional relations. Hence there is a surjective homomorphism from $C_2 \times C_3 \to \<r, s\>$; since both sides have size 6 we have $G \cong C_2 \times C_3$.

Proof 2: By Cauchy's theorem, there exists an element of order 3, say $r$, and $\<r\>$ is a normal subgroup of $G$ (since it has index 2). Hence $G$ has two normal subgroups $N_1, N_2$ of size 2 and 3. By the diamond isomorphism theorem, $G' = N_1 N_2$ is a subgroup of $G$ of size 6 (since its size must be divisible by 2 and 3 but be less than 6). $N_1 \cap N_2 = \{e\}$ by order considerations. Hence by exercise 3.3.7, $G = G' \cong N_1 \times N_2$.

\section*{4.2.11}

In the cycle representation of $\pi(x)$, consider the cycle containing $e$. The cycle must be $(e, x, x^2, \ldots x^{n-1})$ by definition of $n$.

Similarly, consider an arbitrary $g \in G$ with $g \not\in \{e, x, x^2 \ldots x^{n-1}\}$, and consider the cycle containing $g$. The elements $\{g, xg, x^2g, \ldots x^{n-1}g\}$ are all distinct since $x^a g = x^b \iff x^a = x^b$. Hence the cycle in $\pi(x)$ containing $g$ must be $(g, xg, x^2g, \ldots x^{n-1} g)$.

From proposition 3.5.25, $\pi(x)$ is odd $\iff$ the number of cycles of even length is odd $\iff n$ is even and $m$ is odd. For the last biimplication, either $n$ is odd (in which case there are no cycles of even length) or $n$ is even (in which case there are $m$ cycles of even length).

\section*{4.2.12}

Let $H = \{g \in G | \pi(g) \textrm{ is even}\}$. It is easy to check that $H$ is a subgroup of $G$. Let $k \in G$ such that $\pi(k)$ is odd. Then the bijective map $\phi: G \to G$ given by $\phi(g) = kg$ sends $H$ to $G - H$ and $G - H$ to $H$, hence $2|H| = |G|$.

\section*{4.2.13}

Let $G$ be as in the question and $\pi$ be the left regular representation. By Cauchy's theorem, there is an element of order 2, say $s$. Since $|s|$ is even and $|G| / |s| = k$ is odd, by 4.2.11 $\pi(s)$ is an odd permutation. By 4.2.12, $G$ has a subgroup of index 2.

\section*{4.3.13}

Let $G$ be a finite group with $|G| = n$. If $n = 2$ then $G$ has 2 conjugacy classes. Now suppose $G$ has 2 conjugacy classes, and let $G$ act on itself by conjugation; the nonidentity elements form an orbit of size $n-1$. Since the size of the orbit divides the size of the group, $n-1$ divides $n$, hence $n = 2$.

\section*{4.3.23}

The normalizer $N_G(M)$ is a subgroup of $G$ containing $M$, that is $M \le N_G(M) \le G$. By maximality of $M$, either $N_G(M) = M$ or $N_G(M) = G$. 

If $M$ is a maximal subgroup of $G$ and $M$ is not normal, then $N_G(M) \ne G$ hence $N_G(M) = M$. Consider the action of $G$ on $P(G)$ by conjugation. The stabilizer of $M$ is $N_G(M) = M$, hence the orbit of $M$ has size $[G:M]$. Each orb (i.e. element in the image of the action, contained in the orbit) is a subgroup conjugate to $M$ and hence the number of nonidentity elements is $|M|-1$. Hence the number of nonidentity elements of $G$ contained in conjugates of $M$ is at most $[G:M] (|M|-1)$.
\section*{4.3.24}

Since the subgroup lattice is a finite partial order, $G$ is contained in some maximal subgroup $M \ne G$. It suffices to show that $G \ne \cup_{g \in G} gMg\inv$, since $gHg\inv \subseteq gMg\inv$.

The number of nonidentity elements of $G$ contained in conjugates of $M$ is $|G| - 1 \le [G:M] (|M|-1) = \frac{|G|}{|M|}(|M|-1) = |G| - \frac{|G|}{|M|} \le |G| - 2$, a contradiction. Here the first inequality follows from 4.3.23 and the second from $|G| / |M| \ge 2$ since $M$ is a proper subgroup of $G$.

\section*{4.3.26}



\end{document}
