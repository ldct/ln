\listfiles
\documentclass{article}

\usepackage{amsmath}
\usepackage{amssymb}
\usepackage{mathtools}
\usepackage{listings}

\DeclarePairedDelimiter\floor{\lfloor}{\rfloor}
\DeclarePairedDelimiter\ceil{\lceil}{\rceil}
\DeclareMathOperator{\cl}{cl}
\DeclareMathOperator{\E}{E}
\def\Z{\mathbb{Z}}
\def\N{\mathbb{N}}
\def\R{\mathbb{R}}
\def\Q{\mathbb{Q}}
\def\K{\mathbb{K}}
\def\T{\mathbb{T}}
\def\B{\mathcal{B}}
\def\XX{\mathfrak{X}}
\def\YY{\mathfrak{Y}}
\def\AA{\mathfrak{A}}
\def\ZZ{\mathfrak{Z}}
\def\BB{\mathcal{B}}
\def\UU{\mathcal{U}}
\def\MM{\mathcal{M}}
\def\M{\mathfrak{M}}
\def\l{\lambda}
\def\L{\Lambda}
\def\<{\langle}
\def\>{\rangle}
\def\f12{\frac{1}{2}}

\usepackage[a4paper,margin=1in]{geometry}

\setlength{\parindent}{0cm}
\setlength{\parskip}{1em}

\title{HW 1}
\date{}

\begin{document}
\maketitle

\section*{HW 1}

\section*{5.}

Suppose otherwise. Let $e = [1]$; then $e$ is the identity element because for all $k \in \Z$ we have $e*[k] = [1]*[k] = [1*k] = [k]$. Let $x=[0]$ and $x^{-1}=[k]$ for some $k \in \Z$, which exists because of the existence of inverses in a group. By the uniqueness of identities in groups we have $[1] = e = x*x^{-1} = [0]*[k] = [0*k] = [0]$, which means $0 \sim 1$, which means $n$ divides $1$, which is not true for $n>1$.

\section*{7.}

Let $\sim$ be a equivalence relation on real numbers: for real numbers $a, b$ let $a \sim b$ if $b - a$ is an integer. This is reflexive as $0$ is an integer, and transitive as the sum of two integers is an integer.

Let $f(l)$ be the fractional part of $l$. We have $f(l) = l - [l] \ge 0$ because $[l] \le l$ by definition. We have $f(l) = l - [l] < 1$ because otherwise, $[l]+1$ would be an integer less than $l$. Hence $x * y \in R$.

Lemma 1: for a positive real number $a$ we have $[a]+1 = [a+1]$. Proof: $[a]+1 \le a+1$ by adding 1 to both sides of the inequality $[a] \le a$. Suppose there is an integer $t > [a]+1$ which satisfies $t \le a+1$; then because the difference of two distinct integers is at least 1, $[a]+2 \le t \le a+1$ which means $[a] + 1 \le a$, contradicting the definition of $[a]$.

Lemma 2: for a positive real number $a$ and a positive integer $k$ we have $[a]+k = [a+k]$ by induction on $k$.

Lemma 3: for two positive reals $a, b$ we have $a \sim b \implies f(a) = f(b)$. Proof: WLOG $b = a+k$ for a positive integer $k$, then $f(b) = f(a+k) = a+k - [a+k] = a+k - ([a] + k) = a - [a] = f(a)$.

Lemma 4: for a positive real $a, f(a) \sim a$. Proof: their difference is an integer by the defition of $f$.

Identity: $0$ is the identity since for all $x \in G, 0*x = f(x+0) = f(x) = x$.

Inverse: let $x \in G$; then $1-x \in G$ and their group product is $f(1-x+x) = f(1) = 0$.

Commutativity: for $x, y \in G$ we have $x+y = y+x$ hence $f(x+y) = f(y+x)$.

Associativity: for $x, y, z \in G$ we have $x*(y*z) \sim x*(y+z) \sim x+(y+z) = (x+y)+z \sim (x*y)+z \sim (x*y)*z$.

\section*{14.}

I'll write the powers of the elements, represented as integers modulo $36$.

$1; o(1)=1$ 

$-1, 1; o(-1) = 2$

$5, 25, 17, 13, 29, 1; o(5) = 6$

$13, 25, 1; o(13) = 3$

$-13, 25, -1, 13, -25, 1; o(-13) = 6$

$17, 1; o(17) = 1$

\section*{22.}

For all positive integers $k$ we have $(g^{-1}xg)^k = g^{-1}x^kg$ by induction on $k$, with inductive step $(g^{-1}xg)^{k+1} = (g^{-1}xg)^k (g^{-1}xg) = g^{-1}x^kg g^{-1}xg = g^{-1}x^kxg = g^{-1}x^{k+1}g$ and base case $k=1$.

In particular for $k=n$ we have $(g^{-1}xg)^n = g^{-1}x^ng = g^{-1}g = 1$, hence $o(g^{-1}xg) \le 1$. Suppose $o(g^{-1}xg) = k$ with $k < n$; then we have $g^{-1}x^kg = 1 \implies x^k g = g \implies x^k = g g^{-1} = 1$, contradicting that $n$ is the least positive integer such that $x^n = 1$.

We have $o(ab) = o(a^{-1}aba) = o(ba)$.

\section*{31.}

For every $g \in t(G)$ create an edge from $g$ to $g^{-1}$; since $t(G)$ does not contain elements which are their own inverses, each edge points to a different element. Since we have $(g^{-1})^{-1} = g$ this forms a set of bidirectional edges, meaning that $|t(G)|$ is even. Hence $|G - t(G)|$ is even, and since $e \not\in t(G)$ it has at least two elements. Let $x$ be such an element with $x \ne e$. We have $x = x^{-1} \implies x^2 = 1$. Since $x \ne e, o(x) = 2$.

\section*{32.}

Suppose otherwise, and let $x^a = x^b$ with $a < b$ be two equal elements from the list, and let $t = b-a$. We have $t \le n-1$ since $b \le n, 0 \le a$. Then $x^t = x^{b-a} = x^b (x^a)^{-1} = x^b (x^b)^{-1} = 1$, contradicting the fact that $n$ is the least positive integer such that $x^n = 1$.

Suppose $t = |x| > |G|$; then $x^0, x^1, \ldots x^{t-1}$ are all distinct elements of $G$, hence $|G| \ge |\{x^0, x^1, \ldots x^{t-1}\}| = t > |G|$, a contradiction.

\section*{35.}

Let $x^k$ be such an integer power, and let $k = qn + r$ where $0 \le r < n$. We have $x^k = x^{qn+r} = (x^n)^q x^r = 1^q x^r = x^r$ as required.

\end{document}
