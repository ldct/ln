\listfiles
\documentclass{article}

\usepackage{amsmath}
\usepackage{amssymb}
\usepackage{mathtools}
\usepackage{listings}

\DeclarePairedDelimiter\floor{\lfloor}{\rfloor}
\DeclarePairedDelimiter\ceil{\lceil}{\rceil}
\DeclareMathOperator{\cl}{cl}
\DeclareMathOperator{\E}{E}
\def\Z{\mathbb{Z}}
\def\N{\mathbb{N}}
\def\R{\mathbb{R}}
\def\Q{\mathbb{Q}}
\def\K{\mathbb{K}}
\def\T{\mathbb{T}}
\def\B{\mathcal{B}}
\def\XX{\mathfrak{X}}
\def\YY{\mathfrak{Y}}
\def\AA{\mathfrak{A}}
\def\ZZ{\mathfrak{Z}}
\def\BB{\mathcal{B}}
\def\UU{\mathcal{U}}
\def\MM{\mathcal{M}}
\def\M{\mathfrak{M}}
\def\l{\lambda}
\def\L{\Lambda}
\def\<{\langle}
\def\>{\rangle}
\def\f12{\frac{1}{2}}

\usepackage[a4paper,margin=1in]{geometry}

\setlength{\parindent}{0cm}
\setlength{\parskip}{1em}

\title{HW 1}
\date{}

\begin{document}
\maketitle

\section*{HW 1}

\section*{5.}

Suppose otherwise. Let $e = [1]$; then $e$ is the identity element because for all $k \in \Z$ we have $e*[k] = [1]*[k] = [1*k] = [k]$. Let $x=[0]$ and $x^{-1}=[k]$ for some $k \in \Z$, which exists because of the existence of inverses in a group. By the uniqueness of identities in groups we have $[1] = e = x*x^{-1} = [0]*[k] = [0*k] = [0]$, which means $0 \sim 1$, which means $n$ divides $1$, which is not true for $n>1$.

\section*{7.}

Let $f(l)$ be the fractional part of $l$. We have $f(l) = l - [l] \ge 0$ because $[l] \le l$ by definition. We have $f(l) = l - [l] < 1$ because otherwise, $[l]+1$ would be an integer less than $l$. Hence $x * y \in G$.
 
Commutativity: for all $x, y \in G$ we have $x*y = x + y - [x+y] = y+x - [y+x] = y*x$. 

Lemma: $f(l) = r \iff r \in R$ and there exists an integer $t$ such that $r + t = l$. $\implies$ follows because we can take $t = [l]$. $\impliedby$ follows because $f(l) = l-t = r$ where the first equality holds because $t$ cannot be increased (since we have $l - (t+1) = r-1 < 0$).

For associativity we will show that for $a, b, c \in G$ we have $a*(b*c) = f(a + b + c)$; the proof that $(a*b)*c = a + b + c - [a+b+c]$ is similar, and then we have $a*(b*c) = (a*b)*c$.

By our lemma, $a*(b*c) = f(a + b + c) \iff$ there exists an integer $t$ such that $a + b + c = t + a*(b*c)$.

\section*{14.}

I'll write the powers of the elements, represented as integers modulo $36$.

$1; o(1)=1$ 

$-1, 1; o(-1) = 2$

$5, 25, 17, 13, 29, 1; o(5) = 6$

$13, 25, 1; o(13) = 3$

$-13, 25, -1, 13, -25, 1; o(-13) = 6$

$17, 1; o(17) = 1$

\section*{22.}

For all positive integers $k$ we have $(g^{-1}xg)^k = g^{-1}xg g^{-1}xg \ldots g^{-1}xg = g^{-1}x^kg$. In particular for $k=n$ we have $(g^{-1}xg)^k = g^{-1}x^kg = g^{-1}g = 1$, hence $o(g^{-1}xg) \le 1$. Suppose $o(g^{-1}xg) = k$ with $k < n$; then we have $g^{-1}x^kg = 1 \implies x^k g = g \implies x^k = g g^{-1} = 1$, contradicting that $n$ is the least positive integer such that $x^n = 1$.

\section*{31.}

For every $g \in t(G)$ create an edge from $g$ to $g^{-1}$; since $G$ does not contain elements which are their own inverses, each edge points to a different element. Since we have $(g^{-1})^{-1} = g$ this forms a set of bidirectional edges, meaning that $|t(G)|$ is even. Hence $|G - t(G)|$ is even, and since $e \not\in t(G)$ it has at least two elements. Let $x$ be such an element with $x \ne e$. We have $x = x^{-1} \implies x^2 = 1$. Since $x \ne e, o(x) = 2$.

\section*{32.}

Suppose otherwise, and let $x^a = x^b$ with $a < b$ be two equal elements from the list, and let $t = b-a$. We have $t \le n-1$ since $b \le n, 0 \le a$. Then $x^t = x^{b-a} = x^b (x^a)^{-1} = x^b (x^b)^{-1} = 1$, contradicting the fact that $n$ is the least positive integer such that $x^n = 1$.

Suppose $t = |x| > |G|$; then $x^0, x^1, \ldots x^{t-1}$ are all distinct elements of $G$, hence $|G| \ge |\{x^0, x^1, \ldots x^{t-1}\}| = t > |G|$, a contradiction.

\section*{35.}

Let $x^k$ be such an integer power, and let $k = qn + r$ where $0 \le r < n$. We have $x^k = x^{qn+r} = (x^n)^q + x^r = 1^q x^r = x^r$ as required.

\end{document}
