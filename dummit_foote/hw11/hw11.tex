\listfiles
\documentclass{article}

\usepackage{amsmath}
\usepackage{amssymb}
\usepackage{mathtools}
\usepackage{listings}
\usepackage{hyperref}
\usepackage[all,pdf]{xy}
\usepackage{lmodern,amssymb}
\usepackage{multirow}

\DeclarePairedDelimiter\floor{\lfloor}{\rfloor}
\DeclarePairedDelimiter\ceil{\lceil}{\rceil}
\DeclareMathOperator{\cl}{cl}
\DeclareMathOperator{\E}{E}
\def\Z{\mathbb{Z}}
\def\N{\mathbb{N}}
\def\R{\mathbb{R}}
\def\C{\mathbb{C}}
\def\Q{\mathbb{Q}}
\def\K{\mathbb{K}}
\def\T{\mathbb{T}}
\def\B{\mathcal{B}}
\def\XX{\mathfrak{X}}
\def\YY{\mathfrak{Y}}
\def\AA{\mathfrak{A}}
\def\ZZ{\mathfrak{Z}}
\def\BB{\mathcal{B}}
\def\UU{\mathcal{U}}
\def\MM{\mathcal{M}}
\def\M{\mathfrak{M}}
\def\l{\lambda}
\def\L{\Lambda}
\def\<{\langle}
\def\>{\rangle}
\def\f12{\frac{1}{2}}
\def\inv{{-1}}
\def\im{\textrm{im}}
\def\Stab{\textrm{Stab}}
\def\Dic{\textrm{Dic}}
\def\Aut{\textrm{Aut}}
\def\Inn{\textrm{Inn}}


\usepackage[a4paper,margin=1in]{geometry}

\setlength{\parindent}{0cm}
\setlength{\parskip}{1em}

\title{HW 11}
\date{}

\begin{document}
\maketitle

% [From the last problem set.]

% 1. Do Exercise 5.2.14 on p. 167.

% 2. Do Exercise 5.2.15 on p. 167.

% [New problems.]

% 3. Do Exercise 5.4.4 on p. 174.   

% 4. Do Exercise 5.4.5 on p. 174.

% 5. Do Exercise 5.4.14 on p. 174.

% 6. Do Exercise 5.4.18 on p. 174.

\section*{5.2.14a}

The group operation is associative: it is required to prove that for $\alpha, \beta, \gamma \in \hat{G}, (\alpha\beta)\gamma = \alpha(\beta\gamma)$. This follows because $(\alpha(z)\beta(z))\gamma(z) = \alpha(z)(\beta(z)\gamma(z))$ as an equation in $\C$. Similarly, commutativity of multiplication in $\C$ implies commutativity of the group operation in $\hat{G}$.

The identity element is the constant function $\epsilon(z) = 1$ for all $z \in G$, because for any $\alpha \in \hat{G}$, $\alpha(z)\epsilon(z) = \alpha(z)$.

Given $\alpha \in \hat{G}$, the inverse $\alpha^\inv$ is given by $\alpha^\inv(z) = \alpha(z)^\inv$ for all $z \in G$. This is well-defined because $\alpha(z) \neq 0$ for all $z \in G$ (0 is not a root of unity), and $\alpha(z)\alpha(z)^\inv = \alpha(z) \alpha(z)^\inv = 1$, hence $\alpha \alpha^\inv = \epsilon$.

\section*{5.2.14b}

Let $\C'$ be the group of roots of unity in $\C$. Let $j$ be the square root of $-1$ in $\C$.

Write $G = \<x_1\> \times \<x_2\> \ldots \<x_r\>$ and let $n_i$ be the order of $x_i$, and let $\chi_i \in \hat{G}$ be as in the hint. Let $F$ be the function from $G \to \hat{G}$ given by $F(x_1^{e_1}, x_2^{e_2} \ldots x_n^{e_n}) = \chi_1^{e_1} \chi_2^{e_2} \ldots \chi_n^{e_n}$. It is easy to check that $F$ is an injective homomorphism.

We prove that $F$ is surjective. Let $\phi \in \hat{G}$, in other words $\phi : \<x_1\> \times \<x_2\> \ldots \<x_r\> \to \C'$ is a homomorphism. Since the $x_i$ generate $G, \phi$ is uniquely determined by its action on each $x_i$. $\phi(x_i)$ is a root of unity of order dividing $n_i$, so we can write $\phi(x_i) = \chi_i^{e_i}$ for some $e_i$ since $\chi_i(x_i) = e^\frac{2\pi j}{n_i}$ is a primitive $n_i$-th root of unity. Hence $\phi = F(x_1^{e_1}, x_2^{e_2} \ldots x_n^{e_n})$.

\end{document}
