\listfiles
\documentclass{article}

\usepackage{amsmath}
\usepackage{amssymb}
\usepackage{mathtools}
\usepackage{listings}
\usepackage{hyperref}
\usepackage[all,pdf]{xy}
\usepackage{lmodern,amssymb}
\usepackage{multirow}

\DeclarePairedDelimiter\floor{\lfloor}{\rfloor}
\DeclarePairedDelimiter\ceil{\lceil}{\rceil}
\DeclareMathOperator{\cl}{cl}
\DeclareMathOperator{\E}{E}
\def\Z{\mathbb{Z}}
\def\N{\mathbb{N}}
\def\R{\mathbb{R}}
\def\C{\mathbb{C}}
\def\Q{\mathbb{Q}}
\def\K{\mathbb{K}}
\def\T{\mathbb{T}}
\def\B{\mathcal{B}}
\def\XX{\mathfrak{X}}
\def\YY{\mathfrak{Y}}
\def\AA{\mathfrak{A}}
\def\ZZ{\mathfrak{Z}}
\def\BB{\mathcal{B}}
\def\UU{\mathcal{U}}
\def\MM{\mathcal{M}}
\def\M{\mathfrak{M}}
\def\l{\lambda}
\def\L{\Lambda}
\def\<{\langle}
\def\>{\rangle}
\def\f12{\frac{1}{2}}
\def\inv{{-1}}
\def\im{\textrm{im}}
\def\Stab{\textrm{Stab}}
\def\Dic{\textrm{Dic}}
\def\Aut{\textrm{Aut}}
\def\Inn{\textrm{Inn}}


\usepackage[a4paper,margin=1in]{geometry}

\setlength{\parindent}{0cm}
\setlength{\parskip}{1em}

\title{HW 11}
\date{}

\begin{document}
\maketitle

% [New problems.]

% 5. Do Exercise 5.4.14 on p. 174.

% 6. Do Exercise 5.4.18 on p. 174.

\section*{5.2.14a}

The group operation is associative: it is required to prove that for $\alpha, \beta, \gamma \in \hat{G}, (\alpha\beta)\gamma = \alpha(\beta\gamma)$. This follows because $(\alpha(z)\beta(z))\gamma(z) = \alpha(z)(\beta(z)\gamma(z))$ as an equation in $\C$. Similarly, commutativity of multiplication in $\C$ implies commutativity of the group operation in $\hat{G}$.

The identity element is the constant function $\epsilon(z) = 1$ for all $z \in G$, because for any $\alpha \in \hat{G}$, $\alpha(z)\epsilon(z) = \alpha(z)$.

Given $\alpha \in \hat{G}$, the inverse $\alpha^\inv$ is given by $\alpha^\inv(z) = \alpha(z)^\inv$ for all $z \in G$. This is well-defined because $\alpha(z) \neq 0$ for all $z \in G$ (0 is not a root of unity), and $\alpha(z)\alpha(z)^\inv = \alpha(z) \alpha(z)^\inv = 1$, hence $\alpha \alpha^\inv = \epsilon$.

\section*{5.2.14b}

Let $\C'$ be the group of roots of unity in $\C$. Let $j$ be the square root of $-1$ in $\C$.

Write $G = \<x_1\> \times \<x_2\> \ldots \<x_r\>$ and let $n_i$ be the order of $x_i$, and let $\chi_i \in \hat{G}$ be as in the hint. Let $F$ be the function from $G \to \hat{G}$ given by $F(x_1^{e_1}, x_2^{e_2} \ldots x_n^{e_n}) = \chi_1^{e_1} \chi_2^{e_2} \ldots \chi_n^{e_n}$. It is easy to check that $F$ is an injective homomorphism.

We prove that $F$ is surjective. Let $\phi \in \hat{G}$, in other words $\phi : \<x_1\> \times \<x_2\> \ldots \<x_r\> \to \C'$ is a homomorphism. Since the $x_i$ generate $G, \phi$ is uniquely determined by its action on each $x_i$. $\phi(x_i)$ is a root of unity of order dividing $n_i$, so we can write $\phi(x_i) = \chi_i^{e_i}$ for some $e_i$ since $\chi_i(x_i) = e^\frac{2\pi j}{n_i}$ is a primitive $n_i$-th root of unity. Hence $\phi = F(x_1^{e_1}, x_2^{e_2} \ldots x_n^{e_n})$.

\section*{5.2.15}

Let us use the convention that if $G = \<a\> \times \<b\>$ then $a$ has order 8 and $b$ has order 4 (this is always possible since the direct product gives $G \cong C_8 \times C_4$). If $G = \<a\> \times \<b\>$ we have $G = \<a^3\> \times \<b\>$ as well since $\<a\> = \<a^3\>$; similarly we can replace $a$ with $a^3, a^5, a^7$ and we can replace $b$ with $b^3$. 

Since $ord(a) = 8$ we have $a = x^iy^j$ for some $i \in (0, 8)$ with $i$ coprime with $8$ since otherwise the 8 exponents of $x$ in $x^iy^j, x^{2i}y^{2j} \ldots$ are not distinct. We can replace $a$ with some power of a $a'$ such that $a' = xy^k, G \cong \<a'\> \times \<b\>$. Hence we first classify all such $a', b$.

If $b$ is an element of $\<y\>$, we have $b \ne y^2$ (otherwise $b$ has order 2), and some power of $b$ is $y^\inv$, hence some power is $y^-k$. Hence $\<a'\> \times \<b\>$ contains $x$ as well as $y^\inv$, and since $a'$ has order 8 and $b$ has order 4, $G \cong \<a'\> \times \<b\>$. Hence we have $G \cong \<xy^j\> \times \<y\>$ and $G \cong \<xy^j\> \times \<y^3\>$ for $j \in [0, 3]$. Taking odd powers of $xy^j$ we have $G \cong \<x^iy^j\> \times \<y^k\>$ where $i, k$ odd.

If $b$ is not an element of $\<y\>$ we have $a = xy^i, b = x^jy^k$ with $j$ even. Similarly to the above, we replace $b$ with powers of $b$ such that $j = 2$ so $a = xy^i, b = x^2y^k$. Let $l$ be arbitrary; since $y^l$ can be written a product of powers of $a$ and $b$ in a unique way, $y^l$ is some power of $a^{-2}b = y^{k-2i}$. Working modulo 4, $k-2i$ must be congruent to $1$ or $-1$, or $k = 2i +- 1$. Since $2i$ is either 0 or 2 modulo 4, we have $k = 1$ or 3. Hence we have $a = xy^i, b=x^2y^j$ where $i$ is arbitrary and j is even. Taking odd powers of $a$ gives $a = x^hy^i$, $h$ odd. Taking odd powers of $b$ we have $b = x^2y^j$ or $b = x^{-2}y^{-j}$, j even, which is equivalent to $x^6y^j$, $j$ even.

Hence the full list is: either

$G \cong \<x^iy^j\> \times \<y^k\>$ where $i, k$ odd.

$G \cong \<x^hy^i\> \times \<x^iy^j\>$ where $h$ odd, $i = 2,6, j$ even.

\section*{5.4.4}

First, every commutator in $S_n$ is even since $\epsilon(x^\inv y^\inv x y) = \epsilon(xx^\inv) \epsilon(yy^\inv)$ where $\epsilon$ is the sign-counting homomorphism.

Using the identity $[(12),(23)] = (12)(23)(12)(23) = (132)$ we see that any 3-cycles lies in the commutator. Since the 3-cycles in $S_4$ generate $A_4$, $S_4' = A_4$.

Alternatively, from the identity $[(12)(34),(23)] = (14)(23)$ we have $S_4'$ contains the group generated by 3-cycles (a subgroup of order 3) as well as $V_4$ (of order 4); hence it has size at least 12, hence it is $A_4$.

For $A_4'$: looking at the subgroup lattice of $S_4$, the only candidates are the trivial group, $V_4$ and $A_4$. It cannot be the trivial group since $A_4 / A_4'$ is abelian. Also since $A_4 / V_4 = C_3$ is abelian, by 7.4 $A_4' \le V_4$. Hence $A_4' = V_4$.

\section*{5.4.5}

Similar to the above, the 3-cycles generate $A_n$ and are contained in $S_n'$ hence $S_n' = A_n$.

\section*{5.4.14}

We show that $G = D \times U$ as an internal direct product. $D$ and $U$ are both contained in $G$.

% 3. Do Exercise 5.4.4 on p. 174.   



\end{document}
