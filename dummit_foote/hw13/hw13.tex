\listfiles
\documentclass{article}

\usepackage{amsmath}
\usepackage{amssymb}
\usepackage{mathtools}
\usepackage{listings}
\usepackage{hyperref}
\usepackage[all,pdf]{xy}
\usepackage{lmodern,amssymb}
\usepackage{multirow}

\DeclarePairedDelimiter\floor{\lfloor}{\rfloor}
\DeclarePairedDelimiter\ceil{\lceil}{\rceil}
\DeclareMathOperator{\cl}{cl}
\DeclareMathOperator{\E}{E}
\def\Z{\mathbb{Z}}
\def\N{\mathbb{N}}
\def\R{\mathbb{R}}
\def\C{\mathbb{C}}
\def\Q{\mathbb{Q}}
\def\K{\mathbb{K}}
\def\T{\mathbb{T}}
\def\B{\mathcal{B}}
\def\XX{\mathfrak{X}}
\def\YY{\mathfrak{Y}}
\def\AA{\mathfrak{A}}
\def\ZZ{\mathfrak{Z}}
\def\BB{\mathcal{B}}
\def\UU{\mathcal{U}}
\def\MM{\mathcal{M}}
\def\M{\mathfrak{M}}
\def\l{\lambda}
\def\L{\Lambda}
\def\<{\langle}
\def\>{\rangle}
\def\f12{\frac{1}{2}}
\def\inv{{-1}}
\def\im{\textrm{im}}
\def\Stab{\textrm{Stab}}
\def\Dic{\textrm{Dic}}
\def\Aut{\textrm{Aut}}
\def\Inn{\textrm{Inn}}
\def\vphi{\varphi}



\usepackage[a4paper,margin=1in]{geometry}

\setlength{\parindent}{0cm}
\setlength{\parskip}{1em}

\title{HW 13}
\date{}

\begin{document}
\maketitle

\section*{7.3.13}

The map $\varphi : \C \to M_2(\R)$ given by $\varphi(a + bi) = \begin{pmatrix} a & -b \\ b & a \end{pmatrix}$ is an injective ring homomorphism. By proposition 5, this means $\C$ is isomorphic to the image of $\phi$.

$\varphi$ preserves addition: $\varphi$ maps $a + bi + c + di$ to $\begin{pmatrix} a & -b \\ b & a \end{pmatrix} + \begin{pmatrix} c & -d \\ d & c \end{pmatrix} = \begin{pmatrix} a + c & -(b + d) \\ b + d & a + c \end{pmatrix}$, which is $\varphi((a + c) + (b + d)i)$.

$\varphi$ preserves multiplication: $\varphi$ maps $(a + bi)(c + di)$ to $\begin{pmatrix} a & -b \\ b & a \end{pmatrix} \begin{pmatrix} c & -d \\ d & c \end{pmatrix} = \begin{pmatrix} ac - bd & -(ad + bc) \\ ad + bc & ac - bd \end{pmatrix}$, which is $\varphi(ac - bd + (ad + bc)i)$.

$\varphi$ is injective: let $z = a + bi \in \ker \varphi$, then by comparing entries, $a = 1, b = 0$, hence $z = 1$.

\section*{7.3.24a}

We check that $I$ is closed under addition: suppose $a, b \in \vphi^\inv(J)$, then $\vphi(a) = a', \vphi(b') = b'$ for some $a', b' \in J$, hence $\vphi(a + b) = a' + b' \in J$, hence $a + b \in \vphi^\inv(J)$.

We check that $\vphi^\inv(J)$ is closed under left multiplication: suppose $a \in \vphi^\inv(J), r \in R$, then $\vphi(a) = a' \in J$. We have $\vphi(ra) = \vphi(r)a' \in J$ since $J$ is an ideal and $a' \in J$. Hence $ra \in \vphi^\inv(J)$. The proof that $I$ is closed under right multiplication is similar.

Applying it to the inclusion homomorphism $\vphi$, we have $\vphi^\inv(S)$ is an ideal of $R$. $\vphi^\inv(S)$ are all the elements in $R$ that are mapped to an element in $S$, which means all the elements in $R$ which are elements in $S$, which is exactly $R \cap S$.

\section*{7.3.24b}

We check that $\vphi(J)$ is closed under addition. Let $a', b' \in \vphi(J)$, then there exists $a, b \in J$ such that $\vphi(a) = a', \vphi(b) = b'$. $a' + b' = \vphi(a + b) \in \vphi(J)$.

Supposing that $\vphi$ is surjective, we check that $\vphi(J)$ is closed under left multiplication. Let $a' \in \vphi(J), r' \in R$, then there exists $a \in J$ such that $\vphi(a) = a'$. By surjectivity, there exists $r \in R$ such that $\vphi(r) = r'$, hence $r'a' = \vphi(ra) \in \vphi(J)$. The proof for right multiplication is similar.

Surjectivity is required. Otherwise, let $S = \R[x]$ and $R$ be the subring of even polynomials (i.e. the ideal generated by $x^2$), and $\vphi$ be the inclusion map. $R$ is an ideal of $R$ (since it is the whole ring) but the image is not an ideal since $x * x^2 \not\in \vphi(R)$.

\section*{7.3.25}

We can use a standard induction proof of the binomial theorem on $n$, like in this link 
\url{https://proofwiki.org/wiki/Binomial_Theorem}

In the induction step, the identity ${n \choose k} + {n\choose k-1} = {n+1\choose k}$ is used. In $R$, this is interpreted as $1 + 1 \ldots = 1 + 1 \ldots$ where the LHS is ${n \choose k} + {n\choose k-1}$ copies of $1$ and the RHS is ${n+1\choose k}$ copies. The equality follows from associativity of addition in $R$ and the equality in $\Z$.

\section*{7.3.26a}

Let $f$ be the homomorphism. The proof that $f$ preserves addition is the same proof that addition is associative in $\Z$ (e.g. by induction) since  $f(a) + f(b) = (1 + 1 \ldots) + (1 + 1 \ldots) = f(a + b)$ where the dots denote repetition a total of $a$ and $b$ times respectively. $f$ preserves multiplication because $f(a)f(b)$ is $(1 + 1 \ldots)(1 + 1 \ldots) = f(ab)$ which we can prove by induction on $a$ and using the distributivity property of addition. Lastly $f(1) = 1$ by definition.

If $n = 0$, all the elements $1, 1+1, \ldots, 0, -1, -1-1, \ldots$ are distinct, as otherwise we have an equality which we could reduce to the form $1 + 1 + \ldots = 0$ for some finite sum on the left (e.g.: if two positive sums are equal, take their difference). Hence $f$ is injective and has kernel $\{0\}$.

Otherwise, $n > 0$ and $f(n) = 0$. For $x \in \Z$ we can write $x = nq + r$ where $0 \le r < n$ and some integer $q$. Then $f(x) = f(nq + r) = f(nq) + f(r) = f(r) \ne 0$. Hence $f(x) = 0 \iff n \vert x$. 

\section*{7.3.26b}

$\Q$ and $\Z[x]$ have characteristic 0 as they contain $\Z$ as a subring, and the inclusion map (which is injective) coincides with the map $f$.

$n\Z[x]$ is the ideal of multiples of $n$, which means it is the ideal of polynomials whose coefficients are multiples of $n$. We have $f(n) = 0$ since $n$ and $0$ are in the same coset of $n\Z[x]$ since $n \in n\Z[x]$. Hence the characteristic is at most $n$. Conversely, all the elements $f(0), f(1) \ldots f(n-1)$ are distinct, since otherwise if $f(a') = f(b'), a' > b'$ we could take the difference to get $f(a' - b') = 0$ but $a' - b' \not\in n\Z[x]$ since $0 < a' - b' < n$ by construction. Hence the characteristic is $n$.

\section*{7.3.26c}

For $0 < k < n, p \vert {n \choose k} = \frac{p!}{k!(p-k)!}$ as integers since $p$ divides the numerator but none of the factors in the denominator, and by unique factorization in integers. Hence in $R$, we have ${p \choose k} = 0$. The result follows from the binomial theorem for $n = p$.

\section*{7.3.29}

$N(R)$ is closed under left multiplication. Let $r \in R, x \in N(R)$ with $x^n = 0$. Then $(rx)^n = r^n x^n = r^n 0 = 0$ so $rx \in N(R)$. The proof for right multiplication follows because $R$ is commutative.

$N(R)$ is closed under addition. Let $x, y \in N(R)$ with $x^n = y^m = 0$. We can replace the exponents with $N = $ max($n, m$) to get $x^N = y^N = 0$. Now $(x + y)^{2N} = \sum_{k=0}^{2N} {2N \choose k} x^k y^{2N-k} = 0$. For each term either $k \ge N$ or $2N - k \ge N$ as otherwise, their sum would be less than $2N$. Hence, each term is 0 and $x + y \in N(R)$.

\section*{7.3.30}

Let $x \in R/N(R)$ be nilpotent with $x^n = 0$. We can write $x = r + N(R), x^n = r^n + N(R) = 0 + N(R)$. Hence $r^n \in N(R)$, that is there exists $m$ such that ${r^n}^m = 0 = r^{nm}$. Hence $r \in N(R)$ and we have $x = 0 + N(R)$, equivalently $x = 0$.

\section*{7.3.34a}

$I + J$ contains $I$ by taking $j = 0$ in the sum, and similarly it contains $J$. Let $R$ contain both $I$ and $J$. Let $i \in I, j \in J$. Then $i, j \in R$ and hence $i + j \in R$. Since $i, j$ were arbitrary, $I + J \subseteq R$.

\section*{7.3.34b}

Let $x \in IJ$; then $x = ij + i'j' + \ldots$ is a sum of products of $i$ and $j$. Since $I$ is a right ideal, $ij \in I, i'j' \in I$ and so on, hence $x \in I$. Similarly, since $J$ is a left ideal, $x \in J$. Hence $x \in I \cap J$, hence $IJ \subseteq I \cap J$.

\section*{7.3.34c}

We can take $R = Z, I = J = 2\Z$. Then $I \cap J = 2\Z$ but $IJ = 4\Z$ as an element of $IJ$ is a sum of products of two even numbers, hence is divisible by 4.

\section*{7.3.34d}

It suffices to show that $I \cap J \subseteq IJ$. Since $R = I + J$ is unital, we have $1 = i + j$ for some $i \in I, j \in J$.

Let $x \in I \cap J$. Then $x = 1x = ix + jx = ix + xj$. $ix \in IJ$ since $i \in I, x \in J$, and similarly $xj \in IJ$. Hence $x \in IJ$.

The hypothesis that $R$ is unital is necessary. Otherwise with $I = J = R = 2\Z$ we have $R$ is commutative and $I + J = R$ but $I \cap J = 2\Z, IJ = 4\Z$.

Also, $1 \in I + J$ does not imply $1 \in I \wedge 1 \in J$, as we can take $R = Z, I = 2\Z, J = 3\Z$.

\end{document}
