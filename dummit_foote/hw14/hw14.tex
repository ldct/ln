\listfiles
\documentclass{article}

\usepackage{amsmath}
\usepackage{amssymb}
\usepackage{mathtools}
\usepackage{listings}
\usepackage{hyperref}
\usepackage[all,pdf]{xy}
\usepackage{lmodern,amssymb}
\usepackage{multirow}

\DeclarePairedDelimiter\floor{\lfloor}{\rfloor}
\DeclarePairedDelimiter\ceil{\lceil}{\rceil}
\DeclareMathOperator{\cl}{cl}
\DeclareMathOperator{\E}{E}
\def\Z{\mathbb{Z}}
\def\N{\mathbb{N}}
\def\R{\mathbb{R}}
\def\C{\mathbb{C}}
\def\Q{\mathbb{Q}}
\def\K{\mathbb{K}}
\def\T{\mathbb{T}}
\def\B{\mathcal{B}}
\def\XX{\mathfrak{X}}
\def\YY{\mathfrak{Y}}
\def\AA{\mathfrak{A}}
\def\ZZ{\mathfrak{Z}}
\def\BB{\mathcal{B}}
\def\UU{\mathcal{U}}
\def\MM{\mathcal{M}}
\def\M{\mathfrak{M}}
\def\l{\lambda}
\def\L{\Lambda}
\def\<{\langle}
\def\>{\rangle}
\def\f12{\frac{1}{2}}
\def\inv{{-1}}
\def\im{\textrm{im}}
\def\Stab{\textrm{Stab}}
\def\Dic{\textrm{Dic}}
\def\Aut{\textrm{Aut}}
\def\Inn{\textrm{Inn}}
\def\vphi{\varphi}



\usepackage[a4paper,margin=1in]{geometry}

\setlength{\parindent}{0cm}
\setlength{\parskip}{1em}

\title{HW 14}
\date{}

\begin{document}
\maketitle

\section*{1}

Let $R$ be a ring with abelian group $A$. Define a map $\vphi : R \to End(A)$ which maps $r$ to multiplication by $r$, i.e. let $\vphi(r) = \vphi_r$ where $\vphi_r(a) = r a$.

$\vphi$ is a ring homomorphism. It preserves addition: this is equivalent to $\vphi_{r + s} = \vphi_{r} + \vphi_{s}$, or $\forall a \in A, (r+s)a = ra + sa$, which follows from the distributive law in $R$.

Similarly, $\vphi$ preserves multiplication means $\vphi_r \vphi_s = \vphi_{rs}$ where the product on the left denotes function composition in $End(A)$. This means $\forall a \in A, r(sa) = (rs)a$ which follows from associativity of multiplication in $R$.

We show that $\vphi$ is injective by showing its kernel is $\{0\}$. Suppose $\vphi(r) = 0$, that is $r a = 0$ for all $a \in R$. In particular, for $a=1$ we have $r1 = 0$, which means $r=0$.

\section*{7.4.12}

Let $r \in IJ$. Then $r = \sum_{p=0, q=0}^{p=P,q=Q} i_p j_q$ for some $i \in I, j \in J$. Since $I, J$ are f.g., each $i_p = \sum_{r} \alpha_{p,r} a_r$ and each $j_q = \sum_{s} \beta_{q,s} b_s$ hence $r = \sum_{p=0, q=0}^{p=P,q=Q} (\sum_{r} \alpha_{p,r} a_r) (\sum_{s} \beta_{q,s} b_s)$ which is a linear combination of the $\{a_i b_j \}$.

\section*{7.4.15a}

For $p, q \in \mathbb{F}_2[x]$ let $p \sim q$ if $\overline{p} = \overline{q}$, that is $p-q \in \mathbb{F}_2[x]$.

Since $x^2 \sim x+1$, if $p ~ q$ for some $q$ of degree 1 or lower. Proof: otherwise, if $q$ have minimal degree; we can rewrite the highest-order term to get a polynomial of lower degree, a contradiction.

By enumerating possible values of coefficients those polynomials are $\{0, 1, x, x+1\}$. None of these are equivalent to each other as their difference is nonzero and are degree 1 polynomials.

\section*{7.4.15b}

As this is a commutative group of order 4, all of whose nonidentity elements have order 2, it is isomorphic to $V_4$.

\section*{7.4.15c}

$\overline{E}$ is multiplicatively generated by $x$, since the powers are $x, x^2 = x+1, x(x+1) = x^2 + x = x + x + 1 = 1, x$. Hence every nonzero element of $\overline{E}$ has a multiplicative inverse.

\section*{4}

Define the $n$ ideals $R_i = (x-x_i)$ for $x_i \in F$, where $F[x]$ is the ambient ring.

These are pairwise comaximal. Proof: letting the ideals be $R_i = (x-x_i)$ and $R_j = (x-x_j)$, it suffices to show $1 \in R_i + R_j$, since then $F[x] = (1) \subseteq R_i + R_j$. In fact, let $k = (x_j-x_i)^\inv$, which exists since $x_j - x_i \ne 0$, we have $1 = k(x-x_i) + (-k)(x-x_j) \in R_i + R_j$.

Let the ring homomorphism $\vphi_i : F[x] \to F[x]/(x-x_i)$ be the map $p \mapsto p + (x - x_i)$. By the euclidean algorithm on polynomials, $p(x) = q(x) + r(x)(x - x_i)$ where $q$ is constant; substituting $x = x_i$ we get $q = p(x_i) = y_i$.

By the Chinese Remainder Theorem, the map $p \mapsto (p+R_1, \ldots, p+R_n)$ is a surjective ring homomorphism with domain $F[x]$, kernel $R_1 \cdots R_n$ and range (and codomain) $F[x]/R1 \times \cdots \times F[x]/R_n$. Let $Y = (y_1 + R_1, \ldots y_n + R_n)$, and find the unique inverse of $Y$ under this ring homomorphism, which by construction can be represented as $p + R_1 \cdots R_n$ where we have $p(x_i) = y_i$ for all $i$. Furthermore, since $R_1 \cdots R_n$ is generated by the degree $n$ polynomial $(x-x_1) \cdots (x-x_n)$, there is a unique representative $p$ of degree $n-1$ or lower.

\section*{5a}

We consider $R = \Z, a = 2$ then $Z[1/2] = \Z[x]/(2x-1)$. The map $i$ is given by $i : z \mapsto z + (2x-1)$ and it is injective. Suppose $i(z_1) = i(z_2)$; then $z_2 - z_1 = p(x)(2x-1)$ for some $p(x) \in \Z[x]$. Since the LHS is a polynomial of degree $0$ or $-\infty$ and the RHS is a product of $p(x)$ and $2x-1$ which is degree 1, by the additivity of degree under product in $Z[x]$ we have $p(x) = 0$ (it has degree $-\infty$), hence $z_1 = z_2$. (Note: additivity of degree holds because $\Z$ is an integral domain, I'm not sure what happens for general rings).

If we take $R = \Z, a = 0$ then $Z[1/0]$ is $\Z[x]/(-1)$. Since $(-1) = (1) = \Z[x]$, this is the quotient by the whole ring, which results in a ring of size 1 (where $0=1$), hence the map is not injective (it maps an infinite set to a single element).

\section*{5b}

When $a = 0$ we have $R[1/0] = R[x]/(-1)$ and again $R[1/a]$ is isomorphic to the zero ring.

Suppose $R[1/a]$ is the zero ring, that is $0 = 1$ in $R[1/a]$. This means $1 = p(x)(ax-1)$ as an equation in $R[x]$. Hence $R[1/a]$ is the zero ring $\iff$ $ax-1$ is a unit in $R[x]$. However, I can't find a simpler characterisation of this condition.


\end{document}