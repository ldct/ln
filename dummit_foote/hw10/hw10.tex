\listfiles
\documentclass{article}

\usepackage{amsmath}
\usepackage{amssymb}
\usepackage{mathtools}
\usepackage{listings}
\usepackage{hyperref}
\usepackage[all,pdf]{xy}
\usepackage{lmodern,amssymb}
\usepackage{multirow}

\DeclarePairedDelimiter\floor{\lfloor}{\rfloor}
\DeclarePairedDelimiter\ceil{\lceil}{\rceil}
\DeclareMathOperator{\cl}{cl}
\DeclareMathOperator{\E}{E}
\def\Z{\mathbb{Z}}
\def\N{\mathbb{N}}
\def\R{\mathbb{R}}
\def\Q{\mathbb{Q}}
\def\K{\mathbb{K}}
\def\T{\mathbb{T}}
\def\B{\mathcal{B}}
\def\XX{\mathfrak{X}}
\def\YY{\mathfrak{Y}}
\def\AA{\mathfrak{A}}
\def\ZZ{\mathfrak{Z}}
\def\BB{\mathcal{B}}
\def\UU{\mathcal{U}}
\def\MM{\mathcal{M}}
\def\M{\mathfrak{M}}
\def\l{\lambda}
\def\L{\Lambda}
\def\<{\langle}
\def\>{\rangle}
\def\f12{\frac{1}{2}}
\def\inv{{-1}}
\def\im{\textrm{im}}
\def\Stab{\textrm{Stab}}
\def\Dic{\textrm{Dic}}
\def\Aut{\textrm{Aut}}
\def\Inn{\textrm{Inn}}


\usepackage[a4paper,margin=1in]{geometry}

\setlength{\parindent}{0cm}
\setlength{\parskip}{1em}

\title{HW 10}
\date{}

\begin{document}
\maketitle

% qns. why G/Z abelian is not enough?

% 4. Do Exercise 5.1.12 on p. 157.

% 5. Do Exercise 5.2.14 on p. 167.

% 6. Do Exercise 5.2.15 on p. 167.

\section*{4.4.13}

$G$ is partitioned into conjugacy classes where each class has size 1, 7 or 29. The class equation is $203 = |Z(G)| + 7x + 29y$ where $x$ is the number of conjugacy classes of size 7 and $y$ is the number of conjugacy classes of size 29. Since $H$ is a union of conjugacy classes, $H$ cannot contain any element $g$ which belongs in a size-7 class, since then $|H| \ge 8$ (since it contains all the conjugates of $g$ as well as $e$). Similarly it cannot contain any element which belongs in a size-29 class. Hence $H \le Z(G)$.

Now $|Z(G)| \ge 7$, and by Lagrange's theorem $|Z(G)|$ is one of $7, 29, 203$. If $|Z(G)| = 7$ then $G / Z(G)$ is order 29, hence cyclic, hence $Z(G) = G$ (by 3.1.36), a contradiction. Similarly $|Z(G)| \ne 29$. Hence $Z(G) = 203$ and $G$ is abelian.

\section*{4.4.18a}

For $f, g \in G$ let $f \sim g$ if they are conjugates. It suffices to show that $f \sim g \implies \sigma(f) \sim \sigma(g)$, since $\sigma^\inv \in \Aut(G)$.

If $f \sim g$ there exists $x \in G$ such that $f = xgx^\inv$; then $\sigma(f) = \sigma(xgx^\inv) = \sigma(x)\sigma(g)\sigma(x)^\inv$, hence $\sigma(f) \sim \sigma(g)$.

\section*{4.4.18b}

Call a member of $K'$ an involution. Any involution has cycles of length 1 or 2, hence the cycle structure must be $2, 2+2, 2+2+2, \ldots$. Here are the values for $n$ for which the longest cycle type possible is $\#(2+2+2)$:

\begin{tabular}{ |c|c|c|c| } 
    \hline
    n & \#(2) & \#(2+2) & \#(2+2+2) \\
    \hline
    2 & 1 & 0 & 0 \\ 
    3 & 3 & 0 & 0 \\ 
    4 & 6 & 3 & 0 \\ 
    5 & 10 & 15 & 0 \\ 
    6 & 15 & 45 & 15 \\ 
    \hline
    n & ${n \choose 2}$ & $\frac{1}{2!}{n \choose 2}{n-2 \choose 2}$ & $\frac{1}{3!}{n \choose 2}{n-2 \choose 2}{n-4 \choose 2}$ \\
    \hline
\end{tabular}

This completes the proof for $n \le 6$. The general formula is displayed in the last row.

Going from one cell to the one on the right, we multiply by $\frac{1}{2}{n-2 \choose 2}, \frac{1}{3}{n-4 \choose 2}$, etc. This sequence of factors is decreasing, hence a table row is weakly increasing then decreasing (since once a factor becomes less than 1, it will never exceed 1). Hence it suffices to check that $\#(2)$ is less than $\#(2+2)$ and also less than the rightmost nonzero entry in the table.

The first inequality follows since $\frac{1}{2}{n \choose 2}$ is a strictly increasing function for $n > 2$.

For the second inequality, the rightmost nonzero entry is $\frac{n!}{\frac{n}{2}! 2^\frac{n}{2}}$ where the division. The ratio of this to $\#(2)$ is $\frac{(n-2)!}{\frac{n}{2}! 2^{\frac{n}{2} - 1}} = \frac{(n-2)(n-3) \ldots (\frac{n}{2} + 1)}{2^{\frac{n}{2}-1}}$. The number of factors in the top is $\ceil{\frac{n}{2}}-2$ and the number of factors at the bottom is $\floor{\frac{n}{2}} - 1$; these differ by at most 1. For $n \ge 6$, we can thus group them as $\frac{n-2}{p} \frac{n-3}{2} \frac{n-4}{2} \ldots$ where $p$ is 2 or 4. Each factor is greater than 1 for $n \ge 6$.

For $\sigma \in \Aut(S_n)$ since $\sigma(K)$ must be a conjugacy class of size $|K|$, and furthermore $\sigma$ preserves orders, we have $\sigma(K) = K$.

\section*{4.4.18c}

Note that all transpositions are self-inverse. WLOG, we can let $\sigma((1, 2)) = (a, b_2)$. Let $\sigma((2, 3)) = (p, q)$.

Note that $(2, 3)(1, 2)(2, 3) = (1, 3)$, and hence $(p, q)(a, b_2)(p, q) = \sigma((1, 3))$.

If $\{p, q\}$ is disjoint from $\{1, 2\}$ then the HLS is equal to $(a, b_2)$ which violates the injectivity of $\sigma$. Similarly $\{p, q\}$ cannot have two elements in common with $\{a, b_2\}$; hence it has exactly one element in common.

A similar proof shows that $\sigma$ preserves transposition overlap count (if $a, b, c$ are distinct, then $\sigma((a, b))$ and $\sigma((b, c))$ are transpositions $(p, q), (r, s)$ with exactly one element in common, i.e. $|\{p, q, r, s\}| = 3$).

Hence, if $\sigma((1, 2)) = (p, q)$, then $\sigma(1, k)$ has exactly one element in common with $\{p, q\}$ (except for $k=2$). It suffices to show that this is the same element for all $k$. Supposing otherwise, assume WLOG $\sigma((1, 3)) = (p, q')$ and $\sigma((1, 4)) = (p', q)$. Now $(1, 3)$ and $(1, 4)$ have 1 element in common, so $(p, q'), (p', q)$ have one element in common, hence $p' = q'$, so $\sigma((1, 3)) = (p, p'), \sigma((1, 4)) = (q, p')$. Now by a similar overlap-counting argument, $(3, 4)$ must be mapped to $(p, q)$, but this violates the injectivity of $\sigma$.

\section*{4.4.18d}

For arbitrary $1 \le p < q \le n$ we have $(1, q)(1, p)(1, q) = (p, q)$. Hence the given set generates all transpositions, hence all of $S_n$.

Hence any $\sigma \in \Aut(S_n)$ is uniquely determined by its action on $(1, 2), \ldots (1, n)$, hence by the distinct values $a, b_2, \ldots b_n$. There are at most $n!$ such possible values.

The map $f : S_n \to \Aut(S_n)$ which maps $\tau$ to conjugation by $\tau$ is injective since $Z(S_n) = 1$ if $n \ge 3$. Hence there are $n!$ inner automorphisms, which accounts for all $n!$ possible automorphisms.

\section*{4.4.19a}

$|K| \ne |K'|$: this follows by reading off the table in 4.4.18b. Now let $H \le \Aut(S_6)$ be defined as $H = \{\sigma \in \Aut(S_6) : \sigma(K) = K\}$. Let $t_1, \ldots t_{15}$ be the transpositions in $S_6$, and $p_1 \ldots p_{15}$ be the triple transpositions, and let $\sigma \in \Aut(S_6)$. If $\sigma(t_1) = t_k$ for some $k$, then $\sigma \in H$. Otherwise $\sigma(t_1) = p_k$ for some $k$, and hence $\sigma(K) = K'$, and furthermore $\sigma(K') = K$.

If $H$ is equal to $\Aut(S_6)$ then $H$ is index 1. Hence it suffices to show that if there exists some $\tau \in \Aut(S_6) - H$, then $H$ is index 2. For all $\sigma \in \Aut(S_6)$ either $\sigma(K) = K$ in which case $\sigma \in H$ or $\sigma(K) = K'$ in which case $\tau\sigma \in H$. Hence $\Aut(S_6) = H \sqcup \tau H$ and $H$ is of index 2.

\section*{4.4.19b}

By repeating 4.4.18c-d, $|H| = 6!$ and $H = \Inn(S_6)$ (since every inner automorphism belongs to $H$, and there are $n!$ inner automorphisms).

\section*{5.1.12a}

The image of $A$ in $A \times B$ is $\{(a, e) : a \in a\}$ and the image of $A$ in $A * B$ is $A' = \{(a, e)Z : a \in A\}$. Let $f : A \to A'$ be given by $f(a) = (a, e)Z$. This is a homomorphism since $f(a)f(b) = (a, e)(b, e)Z = (ab, e)Z = f(ab)$. This is surjective by definition.

This is injective. Suppose $f(a) = f(b)$, then $(a, e)Z = (b, e)Z$, then $(a, e)\{x_i, y_i^\inv : x_i \in Z_1\} = (b, e)\{x_i, y_i^\inv : x_i : Z_1\}$, then $\{(ax_i, y_i^\inv): x_i : Z_1\} = \{(bx_i, y_i^\inv): x_i : Z_1\}$. There is only a single tuple in both the LHS and the RHS with $e$ as the second argument, with $y_i = x_i = e$. Hence by comparing the first element of that tuple, $ae = be$, hence $a = b$.

The proof that the image of $B$ is isomorphic to $B$ is similar.

The intersection is $I = \{(ax_i, y_i^\inv)Z: a \in A, x_i \in Z_1\} \cap \{(x_i, by_i^\inv)Z: b \in B, x_i \in Z_1\}$. An element of this intersection is of the form $(ax_i, y_i^\inv)Z = (x_j, bx_j^\inv)Z$ for some $a \in A, b \in B, x_i, \in Z_1, y_j \in Z_2$. This means $(ax_i x_j^\inv, y_i^\inv x_jb^\inv) \in Z$; in particular, $a \in Z_1, b \in Z_2$. Hence $I = \{(ax_i, y_i^\inv)Z: a \in Z_1, x_i \in Z_1\} \cap \{(x_i, by_i^\inv)Z: b \in Z_2, x_i \in Z_1\}$, which is central. This is isomorphic to the intersection of the image of $Z_1$ and $Z_2$ in the group $I' = (Z_1 \times Z_2) / Z$ (here $Z$ is understood as a subgroup of $Z_1 \times Z_2$). Hence it suffices to show that the image of $Z_1$ is the entire group. Let $z_1 \in Z_1, z_2 \in Z_2$ be arbitrary; it is required to show that $(z_1, z_2)Z = (a, e)Z$ for some $a \in Z_1$, or equivalently $(az_1^\inv, z_2^\inv) \in Z$. Here we can take $a = z_2' z_1$ where $z_2'$ is the image of $z_2$ under the isomorphism $Z_1 \cong Z_2$.

$|A * B| = |(A \times B) / Z| = |A||B| / |Z_1|$ since $|Z| = |Z_1|$.

\section*{5.1.12b}

In $Z_4 * Q_8$ let $X = xZ, I = iZ, j = jZ, k = kZ$. Let $\phi$ be the set-function $\phi : Z_4 * Q_8 \to Z_4 * Q_8$ be given by $\phi(X) = (x, e)Z, \phi(I) = (e, r)Z, \phi(J) = (x, rs)Z$ where the $Z$'s on the RHS should be understood as the appropriate subgroup of $Z_4 * D_8$. 

Since $\{X, I, J\}$ is a generating set, and the image of these forms a generating set of the RHS as well, it suffices to show that this is a group homomorphism.

By considering $Z_4 * Q_8$ as a subgroup of the direct product, it suffices to check the identities $X^4 = I^4 = J^4 = (IJ)^4 = eZ, IJK = X^2$ under the map. First, note that $\phi(K) = (x, r^2s)$.

This becomes $x^4 = e, r^4 = e, (x, rs)^4 = e, (x, r^2s)^4 = e$ and $(x^2, e) = (x^2, rrsr^2s)$.
\section*{5.1.14}

Let $B = B_1 \times \ldots B_n = \{(b_1 \ldots b_n) : b_i \in B_i\}, g \in G$. Then $g = (g_1, \ldots g_n)$ where $g_i \in A_i$. We have $gB = \{(g_1, \ldots g_n)(b_1, \ldots b_n) : b_i \in B_i\} = \{(g_1b_1, \ldots g_nb_n) : b_i \in B_i\} = \{(b_1'g, \ldots b_n'g) : b_i' \in B_i\} \subseteq Bg$ where in the last equality we have used the fact that $B_i$ is normal in $A_i$. Hence $gB \subseteq Bg$ for all $g$, which implies $B$ is normal in $G$.

Let $\phi : G / B \to (A_1/B_1) \times \ldots (A_n/B_n)$ be the set-function given by $\phi((g_1, \ldots g_n)B) = (g_1B_1, \ldots g_nB_n)$. This is well-defined since if $gB = hB$ then comparing elementwise, $g_iB_i = h_iB_i$. We will prove that $\phi$ is a group isomorphism.

$\phi$ is a group homomorphism: $\phi$ maps the identity $(e_1, \ldots e_n)B$ to $(e_1B_1, \ldots e_nB_n)$, which is the identity in $(A_1/B_1) \times \ldots (A_n/B_n)$. Now $\phi((g_1, \ldots g_n)B(g_1', \ldots g_n')B) = \phi((g_1g_1', \ldots g_ng_n')B) = (g_1g_1'B_1, \ldots g_ng_n'B_n) = (g_1B_1, \ldots g_nB_n)(g_1'B_1, \ldots g_n'B_n) = \phi((g_1, \ldots g_n)B)\phi((g_1', \ldots g_n')B)$. The proof for inverse is similar.

$\phi$ is surjective: for any element in the codomain $y = (a_1B_1, \ldots a_nB_n)$, we can take $g = (a_1, \ldots a_n)$ and $\phi(gB) = y$.

$\phi$ is injective: if $(g_1B_1, \ldots g_nB_n) = (h_1B_1, \ldots h_nB_n)$ then comparing elementwise, $g_iB_i = h_iB_i$. In the domain it is required to prove that $(g_1, \ldots g_n)B = (h_1, \ldots h_n)B$, equivalently $(g_1^\inv, h_1 \ldots g_n^\inv, h_n) \in B$, equivalently $g_i^\inv h_i \in B_i$, which follows from $g_iB_i = h_iB_i$.

\end{document}
