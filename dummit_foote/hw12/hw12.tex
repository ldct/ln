\listfiles
\documentclass{article}

\usepackage{amsmath}
\usepackage{amssymb}
\usepackage{mathtools}
\usepackage{listings}
\usepackage{hyperref}
\usepackage[all,pdf]{xy}
\usepackage{lmodern,amssymb}
\usepackage{multirow}

\DeclarePairedDelimiter\floor{\lfloor}{\rfloor}
\DeclarePairedDelimiter\ceil{\lceil}{\rceil}
\DeclareMathOperator{\cl}{cl}
\DeclareMathOperator{\E}{E}
\def\Z{\mathbb{Z}}
\def\N{\mathbb{N}}
\def\R{\mathbb{R}}
\def\C{\mathbb{C}}
\def\Q{\mathbb{Q}}
\def\K{\mathbb{K}}
\def\T{\mathbb{T}}
\def\B{\mathcal{B}}
\def\XX{\mathfrak{X}}
\def\YY{\mathfrak{Y}}
\def\AA{\mathfrak{A}}
\def\ZZ{\mathfrak{Z}}
\def\BB{\mathcal{B}}
\def\UU{\mathcal{U}}
\def\MM{\mathcal{M}}
\def\M{\mathfrak{M}}
\def\l{\lambda}
\def\L{\Lambda}
\def\<{\langle}
\def\>{\rangle}
\def\f12{\frac{1}{2}}
\def\inv{{-1}}
\def\im{\textrm{im}}
\def\Stab{\textrm{Stab}}
\def\Dic{\textrm{Dic}}
\def\Aut{\textrm{Aut}}
\def\Inn{\textrm{Inn}}


\usepackage[a4paper,margin=1in]{geometry}

\setlength{\parindent}{0cm}
\setlength{\parskip}{1em}

\title{HW 12}
\date{}

\begin{document}
\maketitle

\section*{5.4.14}

We show that $G = D \times U$ as an internal direct product. $D$ and $U$ are both contained in $G$, and every element of $d$ commutes with every element of $u$. Given $g \in G$ whose diagonal elements are $g_0$, we have $g_0 > 0$ since otherwise, the determinant (which is the product of the diagonal elements for an upper triangular matrix) is 0, which contractis $G \subseteq GL_n(F)$. We can write $g = d u$ where $d = g_0 I \in D, u \in U$.

Claim: this decomposition is unique. Suppose $g = d_1 u_1 = d_2 u_2$. Let $d_i = k_i I$ for some scalars $k_i$. Then $k_1 u_1 = k_2 u_2$. The upper left element of the LHS is $k_1$ and that of the RHS is $k_2$, hence $k_1 = k_2$, hence $d_1 = d_2$ and $u_1 = u_2$.
 taking determinants, $|d_1| = |d_2|$ hence they have equal diagonal elements, hence $d_1 = d_2$, hence $u_1 = u_2$.

The map $\phi: H \times K \to G$ given by $(h, k) \mapsto h k$ is a homomorphism. It maps $(e, e)$ to $e$, it maps $(h, k)^\inv = (h^\inv, k^\inv)$ to to $(hk)^\inv = h^\inv k^\inv$, and it maps $(h, k)(h', k') = (hh', kk')$ to $hh'kk' = (hk)(h'k')$. There is a map $G \to H \times K$ given by the unique decomposition. This is an inverse map since the composed map maps $hk \to (h, k) \to hk$, hence $\phi$ is an isomorphism.

\section*{5.5.5a}

By definition, $G = H \rtimes K$ where $H = Z_2 \times Z_2$ and $K = \Aut(H)$. Let $Z_2 = \{e, x, y, xy\}$. Any automorphism of $H$ fixes $e$, hence $K$ is a subgroup of $S_{\{x, y, xy\}}$. In fact, since $Z_2$ is generated by $x, y$ with $x^2 = y^2 = (xy)^2 = e$ which is symmetric in $x, y, xy$, any permutation of $x, y, xy$ is an automorphism of $H$ (e.g. the permutation $(x\ y)$ gives the homomorphism $e \mapsto e, x \mapsto y, y \mapsto x, xy \mapsto xy$). Hence $K = S_{\{x, y, xy\}}$.

$|G| = |H||K| = 4 \times 6 = 24$.

\section*{5.5.5b}

Notation: given $v \in H, \sigma \in K$ let the action of $K$ on $v$ that defines the semidirect product be written $v^\sigma$, and let elements of $K$ be written as $v \sigma$ where $v \in H, \sigma \in K$. Then the multiplication rule for $G$ can be written as $v \sigma_1 w \sigma_2 = vw^{\sigma_1} \sigma_1 \sigma_2$.

Consider $H$ as a subgroup of $K$. There are 4 left-cosets which are $C = \{K, xK, yK, xyK\}$. $G$ acts on $C$ by left multiplication; denote this action by $g \cdot wK$ where $g \in G, w \in H$. This action has a permutation representation $\pi : G \to S_C$. Write $g = v \sigma$. Then $g \cdot wK = v \sigma \cdot wK = vw^\sigma \sigma K = vw^\sigma K$. 

Notation: when writing elements of $\im \pi$ as permutations, we shall replace $vK$ with $v$.

Considering the elements $g$ for which $v = e$, we see that $g \cdot wK = w^\sigma K$. Hence $\im \pi$ contains $S_{\{x, y, xy\}}$.

Considering the elements of $g$ for which $\sigma$ is the identity automorphism, we see that $g \cdot wK = vwK$. Hence $\im \pi$ contains $(e\ x)(y\ xy)$ as the image of $x$, $(e\ y)(x\ xy)$ as the image of $y$, and $(e\ xy)(x\ y)$ as the image of $xy$ (these are the permutations that correspond to the left regular representation of the action of $H$ on itself by left multiplication)

We know $\im \pi$ contains $(x\ y)$. Conjugating this by $(e\ x)(y\ xy)$ we see that it contains $(e\ xy)$. Conjugating this by elements of $S_{\{x, y, xy\}}$ we see that it contains $(e\ x)$ and $(e\ y)$. Hence $\im \pi$ contains all the tranpositions of $S_C$, hence $\im \pi = S_C$, which means there is a surjective homomorphism $G \to S_4$; since $|G| = |S_4|$, this is an isomorphism.


\section*{5.5.16}

I will use $C_k$ for the cyclic group of order $k$.

Considering $C_8$ as the additive group of $\Z / 8 \Z$, every automorphism of $C_8$ is multiplication by a unit of $\Z / 8 \Z$, i.e. the automorphisms are precisely multiplication by 1, 3, 5, or 7. Every homomorphism $C_2 \to C_8$ is determined by the image of the nonidentity element of $C_2$, and an automorphism of $C_8$ is a valid target if it is self-inverse. It is easy to check that each of those four automorphisms is self-inverse (since $-1, 3, -3$ all square to 1 modulo 8).

Multiplication by one: this is the semidirect product $C_2 \rtimes C_8$ where the action of $C_2$ on $C_8$ is trivial, hence it is the direct product $C_2 \times C_8$.

Multiplication by $7 = -1$: this is the semidirect product $C_2 \rtimes C_8$ where the action of $C_2$ on $C_8$ is inversion, hence it is the dihedral group.

Multiplication by $3$: use exponential notation for the action. The multiplication rule is $\sigma_1 \tau_1 \sigma_2 \tau_2 = \sigma_1 \sigma_2 \tau_1^{\sigma_2} \tau_2$. Let $\sigma_1 = e, \tau_2 = e$ and let $\tau, \sigma$ be generators of $C_8$ and $C_2$. The multiplication rule simplifies to $\tau \sigma = \sigma^\tau \tau = \sigma^3 \tau$. Hence $\tau \sigma^3 = \sigma^3 \tau \sigma^2 = \sigma^6 \tau \sigma = \sigma^9 \tau = \sigma \tau$, hence this group has the presentation $\< \sigma, \tau \mid \sigma^8 = \tau^2 = e, \sigma \tau = \tau \sigma^3 \>$ which is the same presentation as 2.5.11.

Multiplication by 5: let $u, v$ be generators of $C_2$ and $C_8$; similar to the above, we have $u v = v^5 u$. Hence $u v^5 = v^5 u v^4 = v^{10} u v^3 = \ldots = v^{25} u = vu$ which is the same presentation in 2.5.14.

\section*{5.5.18}

Let $H$ be a group; by Cayley's theorem it is a subgroup of $S_H$, namely the subgroup of permutations $P = \{ \pi_h: h' \mapsto hh' \mid h \in H\}$. Take $G = N_{S_H}(H)$; then $H$ is a normal subgroup of $G$ by construction. Let $\sigma$ be an automorphism of $H$; in particular $\sigma$ is a permutation of $H$ so we can treat $\sigma$ as an element of $S_H$. The conjugation $\pi_h^\sigma$ is the map $\sigma \circ \pi_h \circ \sigma^\inv$ which maps $h'$ to $\sigma(h\sigma^\inv(h')) = \sigma(h) h'$ hence it is the map $\pi_{\sigma(h)}$. Hence conjugation by $\sigma$ is an automorphism of $P$ (this also shows $\sigma \in G$).

\section*{5.5.22a}

We apply the recognition theorem (theorem 12):

$U \trianglelefteq G$: not sure how to prove this

$U \cap D = \{I\}$: let $\mathcal{M}$ be in the intersection. $\mathcal{M}$ is a diagonal matrix, and all of its diagonal elements are 1, hence it is the identity matrix.

$UD = G$: note that diagonal matrices in $GL_n(F)$ are invertible, with

\[ 
\begin{pmatrix} a & 0 & \ldots \\ 0 & b & \ldots \\ & & \ddots \end{pmatrix}^\inv =
\begin{pmatrix} a^\inv & 0 & \ldots \\ 0 & b^\inv & \ldots \\ & & \ddots \end{pmatrix}
\]

where every diagonal element is nonzero, since otherwise the determinant would be 0. Given an arbitrary matrix $\mathcal{G} = \begin{pmatrix} a & \ldots & \ldots \\ 0 & b & \ldots \\ & & \ddots \end{pmatrix}$ let $\mathcal{D} = \begin{pmatrix} a & 0 & \ldots \\ 0 & b & \ldots \\ & & \ddots \end{pmatrix}$; we have $\mathcal{G} \mathcal{D}^\inv$ is an upper triangular matrix with diagonal elements 1; call this matrix $\mathcal{U}$. Then $\mathcal{G} = \mathcal{U} \mathcal{D}$.
\section*{5.5.22b}

Elements of $U$ are of the form $\mathcal{U} = \begin{pmatrix} 1 & x \\ 0 & 1 \end{pmatrix}$ where we can take $x \in F$ to be the image of the isomorphism. Similarly elements of $D$ are of the form $\mathcal{D} = \begin{pmatrix} a & 0 \\ 0 & b \end{pmatrix}$ where $a, b \in F^\times$.

The action is given by conjugation of $\mathcal{U}$ by $\mathcal{D}$; we have $\mathcal{D} \mathcal{U} \mathcal{D}^\inv = \begin{pmatrix} 1 & a^\inv x b \\ 0 & 1 \end{pmatrix}$. Hence the action is given by $(a, b) \circ x =  a^\inv b x$.


\end{document}
