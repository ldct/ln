\listfiles
\documentclass{article}

\usepackage[pdftex]{graphicx}
\usepackage{amsmath}
\usepackage{amssymb}

\usepackage[a4paper,margin=1in]{geometry}

\newcommand{\half}{\frac{1}{2}}
\newcommand{\<}{\langle}
\renewcommand{\>}{\rangle}

\title{Quantum Mechanics - Griffiths, David J}
\date{}

\begin{document}
\maketitle

\section{The Wave Function}

\subsection{}
\subsection{}
\subsection{Gaussian distribution}
``Consider the Gaussian distribution..." 
\\ \\
a) $\sqrt{\frac{\lambda}{\pi}}$ \\
b) $\<x\> = a, \<x^2\> = \frac{1}{2\lambda} + a^2, \sigma^2 = \frac{1}{2\lambda}$ \\
c) a smooth gentle hump centered at $a$
\subsection{}
a) $A = \frac{2}{b}$ \\
b) a sharp concave up peak \\
c) ??
\subsection{Delta potential}
a) $A = \sqrt{\lambda}$ \\
b) $\<x\> = 0, \<x^2\> = \frac{1}{2\lambda^2}$ \\
c) $\sigma = \frac{\sqrt 2}{2} \frac{1}{\lambda}, \Pr(|x| > \sigma) = e^{-\sqrt{2}}$
\subsection{}
\subsection{}
\subsection{}
\subsection{}
a) $A^2 = \sqrt{\frac{2am}{\pi \hbar}}$ \\
b) $V = 2a^2mx^2 \\
c) \<x\> = 0, \<x^2\> = \frac{\hbar}{4am}, \<p\> = 0, \<p^2\> = am\hbar$ \\
d) $\sigma_x^2 \sigma_p^2 = \frac{\hbar^2}{4}$
\subsection{}
\subsection{}
\subsection{}
\subsection{}
\subsection{}
a) ?? b) 0
\subsection{Unstable particle}
``Suppose you wanted to describe an unstable particle..."
\\ \\
a) ?? b) $P = P_0 e^{-(2\Gamma / \hbar) t}$
\subsection{}
Done
\subsection{}
a) $A^2 = \frac{15}{16 a^5}$ b) $\<x\> = 0$ c) $\<p\> = 0$ d) $\<x^2\> = \frac{A^2 a^7 16}{105}$ e) $\<p^2\> = \frac{8}{3} \hbar^2 A^2 a^4$ f,g,h) $\sigma_x^2 \sigma_p^2 = \hbar^2 \frac{5}{2}$

\section{The time-independent Schr\"odinger equation}

\subsection{}
\subsection{}
\subsection{}
Done
\subsection{}
``Calculate $\<x\>, \<x^2\>,...$ for the $n$th stationary state..."
\\ \\
$\<x\> = a/2$ \\
$\<x^2\> = a^2 \left( \frac{1}{3} + \frac{1}{2n\pi} \right)$ \\
$\<p\> = 0$ \\
$\<p^2\> = \frac{\hbar^2 n^2 \pi^2}{a^2} $ \\
$\sigma_x^2 = a^2 \left( \frac{1}{12} + \frac{1}{2n\pi} \right) $ \\
$\sigma_x^2 \sigma_p^2 = \hbar^2 \pi^2 \left( \frac{n^2}{12} + \frac{n}{2\pi} \right) $
\subsection{}
``A particle in the infinite square well has as its initial wave function an even mixture of the first two..."
\\ \\
a) $A = \frac{\sqrt 2}{2}$ \\
b) $\psi(x,t) = \frac{\sqrt a}{a} \left(\sin(\frac{\pi x}{a}) e^{-i\pi^2\hbar / 2ma^2} + \sin(\frac{2\pi x}{a})e^{-4i\pi^2\hbar t / 2ma^2}\right)$,
$|\psi|^2 = \frac{1}{a} \left( \sin^2\frac{\pi x}{a} + \sin^2\frac{2\pi x}{a} + 2\sin\frac{\pi x}{a}\sin\frac{2\pi x}{a}\cos\frac{3\pi^2\hbar}{2ma^2}t  \right)$ \\
c) $\<x\> = \frac{a}{2} - \frac{16a}{9\pi^2} \cos 3\omega t$ \\
d)? e)?
\subsection{}
\subsection{}
$\<x\> = \frac{a}{2}$, 
$\<x^2\> = \frac{2}{7} a^2$,
$\sigma_x^2 = \frac{5}{14} a^2$ \\
$\<p\> = 0$, 
$\<p^2\> = \frac{10 \hbar^2}{a^2}$ \\

\end{document}
