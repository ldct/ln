\listfiles
\documentclass{article}

\usepackage[pdftex]{graphicx}
\usepackage{amsmath}
\usepackage{amssymb}

\usepackage[a4paper,margin=1in]{geometry}

\newcommand{\half}{\frac{1}{2}}
\newcommand{\<}{\langle}
\renewcommand{\>}{\rangle}

\setlength\parindent{0pt}
\setlength\parskip{6pt}

\title{}
\date{}

\begin{document}
\maketitle

\section{APhO 2004 q2}

An optical fiber consits of a cylindrical core of radius a, made of a transparent material with refractive index gradually varying from $n=n_1$ on the axis to $n = n_2$ at a distance $a$ from the axis, according to $n(x) = n_1\sqrt{1-\alpha^2x^2}$ where $x$ is the distance from the core axis and $\alpha$ is a constant. The core is surrounded by a cladding with constant refractiion index $n_2$. Outside the fiber is air of refractive index $n_0$.

Let the $Oz$ be the axis of the fiber, with $O$ the center of the fiber end. A monochromatic light ray enters the fiber at $O$ with an incident angle $\theta_i$ in the $xOz$ plane.

a) Show that at each point on the trajectory of light in the fiber, the relation $n\cos\theta = C$ is satisfied, where $\theta$ is the angle between the light ray and the $Oz$ axis. Find $C$ in tertms of $n_1$ and $\theta_i$.

b) Derive an expresion for $x' = \frac{dx}{dz} = \tan\theta$. Express $\alpha$ in terms of $n_1, n_2$ and $a$. Find also an expression for $x''$.

c) Find the equation of the trajectory of light in the fiber, $x(z)$.

d) Sketch one full period of the trajectories of light rays entering the fiber under 2 different incident angles $\theta_i$.

\section{APhO 2003 q2}

In a Fiber-Optic Gyroscope, the phase shift caused by 2 coherent beams of light sent around a rotating ring of optical fiber in opposite directions can be used to determine the angular speed of the ring.

A light wave of wavelength $\lambda$ enters a circular optical fiber light path of radius $R$ at point $P$ on the rotating platform with a uniform angular speed $\Omega$ in the clockwise direction. Here the wave is split into two waves which travel in the CW and CCW directions through the ring. The refractive index of the optical fiber material is $\mu$. Assume that the light travels in a smooth circular path of radius $R$.

a) Find the time difference $\delta t = t_{CW} - t_{CCW}$, where $t_{CW}$ and $t_{CCW}$ are the round-trip transit times of the CW and CCW beams respectively. Assume that $(R\Omega)^2 << c^2$.

b) Find the optical path difference, $\Delta L$, for the CW and CCW beams after completing one round-trip within the rotating ring.

c) The measurement could be amplified by increasing the number of turns in the fiber-optic coil, $N$. Find the phase difference $\delta\theta$ of the beams after completing the turns.

\section{APhO 2007 q2}

For a material with both negative $\epsilon_r$ and $\mu_r$, when a light wave propagates forward a distance $\Delta$, the phase of the light wave will decrease, rather than increase $\sqrt{\epsilon_r \mu_r}k\Delta$ as what happens in a conventional medium with positive $\epsilon_r$ and $\mu_r$, where k iis the wave vector of the light.

Assume that for air, the relative permittivity and permeability are equal to 1.

a) Assume that a light beam strikes from air the surface of such an unusual material with $\epsilon_r <0$ and $\mu_r < 0$. Prove that the depicted direction of the refracted light beam is reasonable, and find the relationship between the refraction angle $\theta_r$ and incidence angle $\theta_i$.

b) Find the relationship between $\theta_r$ and $\theta_i$ if the beam travels from the unusual material to air.

c) An infinitely long cylinder of radius $R$, made of an unusual optical material with $\epsilon_r = \mu_r = -1$, is placed in air as shown. Suppose a laser source located on the $X$ axis with its position described by the coordinate $x$, emits a narrow laser light along the $Y$ direction. Find the range of $x$, for which the light signal emitted from the light source cannot reach the infinite receiving plane on the other side of the cylinder.

\section{IPhO 1972 q4}

A thin planoconvex lens with diameter $2r$, curvature radius $R$ and refractive index $n_0$ iis positioned so that on its left side is air $n_1 = 1$ and on its right side is a transparent medium with refractive index $n_2 \neq 1$. The convex face of the lens is directed towards the air. In the air, at a distance $s_1$ from the lens, measured on the principal optic axis, there is a point source of monochromatic light.

a) Demonstrate, using a paraxial approximation, the relation $f_1 / s_1 + f_2 / s_2 = 1$, where $s_2$ is the distance of the image from the lens, $f_1$ and $f_2$ are the focal distances of the lens in air and the medium with refractive index $n_2$ respetcively.

b) The lens is cut perpendicular to its plane face in two equal parts. These parts are moved a distance $\delta << r$ apart. Suppose now that the setup is in air, such that $n_2 = n_1 = 1$. On the right side of the lens is a screen at distance $d$, which is parallel to the plane face of the lens. Find the number $N$ of interference fringes on the screen as a function of the wavelength of incident light.

\end{document}
