\documentclass[letterpaper]{article}

\usepackage{geometry}
\usepackage{amsmath}
\usepackage{tikz}
\usepackage{pgfplots}
\usepackage{titlesec}


\setlength\parindent{0em}


\begin{document}

\section{Arithmetic and Geometric Progressions}
\section{Summation of Series}

\subsection{Sigma Notation}

\begin{align*}
\sum_{T=1}^n T_r = T_1 + T_2 + T_3 + \ldots + T_n
\end{align*}

Number of terms = $m - n + 1$

Express the following in sigma notation

\begin{enumerate}
\item $1 - \cfrac{1}{2} + \cfrac{1}{3} - \cfrac{1}{4} + \ldots$
\item $3 + 5 + 7 + 9 + ... + 41$
\item $1 \times 3 - 2 \times 5 + 3 \times 7 - 4 \times 9 + 5 \times 11$
\end{enumerate}

Write down the first 3 terms of the following sums

\begin{enumerate}
\item $\sum_{k=1}^{100} (3k - 1)$
\item $\sum_{k=1}^{40} (2 r^2)$
\item $\sum_{k=3}^{10} (r-1)(2r + 1)$
\end{enumerate}

\subsection{Basic Properties of Sigma}

\begin{align*}
\sum_{r=1}^n kT_r &= k\sum_{r=1}^n T_r \\
k\sum_{r=1}^n (T_r + G_r) &= \sum_{r=1}^n T_r + \sum_{r=1}^n G_r \\
\sum_{r=m}^n T_r &= \sum_{r=1}^n T_r - \sum_{r=1}^{m-1} T_r
\end{align*}

True or False?

\begin{enumerate}
\item $\sum_{r=1}^{100} (2r + 1) = \sum_{r=1}^{100} (2r) + 1$
\item $\sum_{n=1}^{100} (2n + 1) = \sum_{m=1}^{100} (2n + 1)$
\item $\sum_{n=1}^{100} a_n = \sum_{n=0}^{99} T_{n+1}$
\item $\sum_{m=1}^{100} m^2 = \sum_{m=0}^{100} m^2$
\item $\sum_{r=1}^{100} a = 100a$
\item $\sum_{n=1}^{100} (2n + 1) = \sum_{m=2}^{101} (2m - 1)$
\item $\sum_{m=1}^k (m + 2) = \sum_{m = 0}^k m + \sum_{m=0}^k 2$
\end{enumerate}

\subsection{Basic Formulas}

\begin{align*}
\sum_{r=1}^n r &= \cfrac{1}{2} n (n + 1) \\
\sum_{r=1}^n r^2 &= \cfrac{1}{6} n (2n + 1) \\
\sum_{r=1}^n r^3 &= \cfrac{1}{4} n^2 (n+1)^2
\end{align*}

\begin{enumerate}
\item Evaluate $\sum_{r=1}^n (r+2)(2r - 1)$ in terms of $n$.

\item Find $\sum_{k=0}^n (2n + 1 - 2k)$ in terms of $n$.

\item Find an expression, in simplified form, for $\sum_{r = n+1}^{2n} (2r - 1)^2$.
\end{enumerate}

\subsection{Method of difference}

If a general term $T_r$ can be expressed as $G_{r+1} - G_r$, then

\begin{align*}
\sum_{r=1}^n T_r &= \sum_{r=1}^n (G_{r+1} - G_r) \\
&= G_n - G_0
\end{align*}

\begin{enumerate}
\item Evaluate $\sum_{r=1}^{100} \cfrac{1}{r(r+1)}$

\item Find the partial fractions of $\cfrac{1}{(2x - 1)(2x + 1))}$. Hence, find the sum of $\cfrac{1}{1 \times 3} + \cfrac{1}{3 \times 5} + \ldots + \cfrac{1}{(2n - 1)(2n + 1)}$

\item Express $\cfrac{2}{y(y+1)(y+2)}$ in partial fractions. Using this result, show that

\begin{align*}
\sum_{r=1}^N \cfrac{1}{r(r+1)(r+2)} = \cfrac{1}{4} - \cfrac{1}{2(N+1)(N+2)}
\end{align*}

\item By considering $\cfrac{1}{1 + a^{n-1}} - \cfrac{1}{1 + a^n}$ or otherwise, show that $\sum_{n=1}^N \cfrac{a^{n-1}}{(1 + a^{n-1})(1 + a^n)} = \cfrac{a^N - 1}{2(a-1)(a^N + 1)}$, where $a$ is positive and $a \neq 1$. Deduce that $\sum_{n=1}^N \cfrac{2^n}{(1 + 2^{n-1})(1 + 2^n)} < 1$.

\item Express $\cfrac{2}{n(n+1)(n+2)}$ in partial fractions. Hence, evaluate $\sum_{k=1}^{99} \cfrac{1}{k(k+1)(k+2)}$

\item Expand $n^2 - (n-1)^2 e$. Hence or otherwise, prove that $\sum_{n=1}^{N} e^n \left[(1-e)n^2 + 2ne - e\right] = N^2 e^N$.

\item Find $\sum_{r=2}^n \left[r - 2 -2n\left(\cfrac{1}{r-1} - \cfrac{1}{r}\right)\right]$ in terms of $n$.

\item Show that $\cfrac{1}{n^2 - n + 2} - \cfrac{1}{n^2 + n + 2} = \cfrac{2n}{n^4 + 3n^2 + 4}$. Find an expression in terms of $N$ for the sum $S_N$ where $S_N = \sum_{n=1}^N \cfrac{n}{n^4 + 3n^2 + 4}$. Deduce that $S_N < \cfrac{1}{4}$.
\end{enumerate}

\section{Binomial Theorem}

\subsection{Permutation and Combination}

Permutation: Arrangement o things. Combination: Selection of things (order not important)

Multiplication rule: If a work can be done in $m$ ways and another in another $n$ ways, both operations can be done $m \times n$ ways.

\begin{align*}
P_n &= n! \\
^nC_r &= \cfrac{n!}{r!(n-r)!}
\end{align*}

\begin{enumerate}
\item How many ways can the 3 letters A, B and C be arranged in a row?

\item A committee of 5 is to be selected from 10 boys and 10 girls. How many different selections are there if the committee must contain a) boys only b) 2 girls
\end{enumerate}

\subsection{Binomial Theorem}

\begin{align*}
(a + b)^n = \sum_{r=0}^n {n \choose r} a^r b^{n-r}
\end{align*}

\begin{enumerate}
\item Expand $(2x + 1)^5$

\item Expand $(2x^4 - y^2)^3$

\item Find the 5th term of the expansion of $(3x - 4)^{12}$

\item Find the coefficients of $a^4b^3$ and $a^2b^5$ in the expansion of $(2a - 3b)^7$

\item Find the coefficient of $x^5$ and the term independent of $x$ in the binomial expansion $\left(\cfrac{x^2}{2} - \cfrac{3}{x^3}\right)$
\end{enumerate}

\subsection{Binomial Series}

\begin{align*}
(1 + f(x))^n = [1 + nf(x) + \cfrac{n(n-1)}{2!}[f(x)]^2 + \cfrac{n(n-1)(n-2)}{3!} [f(x)]^3 + \ldots + \cfrac{n(n-1)\ldots(n-r+1)}{r!} [f(x)]^r]
\end{align*}

\begin{enumerate}
\item Expand $(3 - 4x)^{-2}$ as a series of ascending powers of $x$ up to and including the term in $x^3$. State the range of values of $x$ for which the expansion is valid.

\item Obtain the first 5 terms in the expansion of $(x^2 + 3x^3)^{\frac{1}{2}}$ in descending powers of $x$. State the range of values of $x$ for which this expansion is valid.

\item Express $f(x) = \cfrac{x-9}{(x-1)(x+3)}$ in partial fractions. Hence or otherwise, obtain $f(x)$ as a series expansion in asceding powers of $x$ as far as the term in $x^3$, given that $-1 < x < 1$. Find also the coefficient of $x^n$ in this expansion, where $n > 1$.

\item Find the possible values of $a$ and $b$ if the expansion in ascending powers of $x$ up to the term in $x^2$ of $\cfrac{\sqrt{1-ax}}{1 + bx}$ is $1 - \cfrac{9}{2} x^2$. With these values of $a$ and $b$, state the set of values of $x$ for which the expansion is valid.

\item Find the coefficient of $x^3$ in the expansion of $\left(1 - \cfrac{x}{2}\right)^{10}$.

\item Given that $x > \cfrac{1}{2}$, obtain the first 3 terms in th eseries expansion of $(x + 2x^2)^{\frac{1}{2}}$ in descending powers of $x$.

\item Given that $|x|>1$, find the expansion of $\sqrt{x + x^2}$ up to and including the 4th nonzero term. Hence, by using a suitable substitution, find an approximation of $\sqrt{12}$ in the form $\cfrac{p}{q}$, where $p$ and $q$ are integers in its lowest terms.

\item Find the values of $m$ if the constant term in the expansion of $\left(1 + 2x^2 - \cfrac{m}{x^4}\right)^6$ is 181.

\item The first three terms in the series expansion of $(a+x)^{-b}$, where $a$ is $a$ real constant and $b>0$, in ascending powers of $x$ are $\cfrac{1}{9} - \cfrac{2x}{27} + \cfrac{x^2}{27} + \ldots$. Find the values of $a$ and $b$. State the range of values of $x$ for which the expansion is valid.
\end{enumerate}

\section{Mathematical Induction}

\subsection{The Principle of Mathematical Induction}

Let $P(n)$ be a statement about the positive integer $n$. Suppose that $P(1)$ is true and 

For any natural number $k$, if $P(k)$ is true then $P(k+1)$ is also true, then $P(n)$ is true for all natural numbers $n$.

\begin{enumerate}
\item Prove by mathematical induction that $\cfrac{1}{1 \times 2} + \cfrac{1}{2 \times 3} + \ldots + \cfrac{1}{n \times (n + 1)} = \cfrac{n}{n + 1}$.

\item A sequence of positive integers $u_1, u_2, u_3 \ldots$ is defined by the relation

\begin{align*}
u_{n+1} = \cfrac{5u_n + 4}{u_n + 2}
\end{align*}

Where $u_1 = 1$. Prove by mathematical induction that $u_n < 4$ for all $n > 1$.

\item Prove that

\begin{align*}
\cos(a) \cos(2a) \cos(4a) \ldots \cos(2^n a) = \cfrac{\sin(2^{n+1} a)}{2^{n+1 \sin(a)}}
\end{align*}

\item Prove by mathematical induction that $3^{4n-2} + 17^n + 22$ is divisible by 16 for every positive integer $n$.

\begin{align*}
1^2 &= \cfrac{1 \times 2 \times 3}{6} \\
1^2 + 3^2 &= \cfrac{3 \times 4 \times 5}{6} \\
1^2 + 3^2 + 5^2 &= \cfrac{5 \times 6 \times 7}{6}
\end{align*}

\item Write down the fourth row. Make a conjecture on a formula on the sum of squares of the first $n$ od dpositive integers. Prove by mathematical induction that the formula is correct.

\item Suppose you have three posts and a stack of $n$ disks, initially placed on one post with the largest disk on the bottom and each disk above it is smaller than the disk below. A legal move involves taking the top disk from one post and moving it so that it becomes the top disk on another post, but every move must place a disk either on an empty post, or on top of a disk larger than itself. Show that for every $n$, it is possible to move all the disks to a different post in finite steps. How many moves are required for an initial stack of $n$ disks?
\end{enumerate}

\subsection{Fallacies}

\subsection{More examples}

\begin{enumerate}
\item Prove by mathematical induction that $5$ is a factor of $6^n - 1$ for all natural numbers $n$.

\item Let $u_n$ denote the number of dots that make up the $nth$ hexagon with side length $n$. Then $u_n = 3(n+1)^2 - 3(n+1) + 1$ for all $n$. Let $S_n = \sum_{r=1}^n u_r$. Find the values of $S_1, S_2, S_3$ and $S_4$. Make a conjecture for the formula $S_n$. Prove by induction your formula $S_n$.

\item Prove by induction that $\cfrac{2}{3!} + \cfrac{2 \times 2^2}{4!} + \cfrac{3 \times 2^3}{5!} + \ldots + \cfrac{n \times 2^n}{(n + 2)!} = 1 - \cfrac{2^{n+1}}{(n + 2)!}$ for all $n \in Z$. Use your result to find an expression in terms of $n$ for $\sum_{r=n}^{2n} \cfrac{r 2^r}{(r + 2)!}$

\item The $rth$ term of of a sequence is give by $u_r = \cfrac{1}{(2r)^2 - 1}$ for $r=1,2,3\ldots$ Write down the first four terms of the sequence, and hence state the values of $\sum_{r=1}^n u_r$ for $n=1,2,3,4$. Make a conjecture for the formula for $\sum_{r=1}^n u_r$ in terms of $n$ and prove the formula by induction.
\end{enumerate}
\end{document}
