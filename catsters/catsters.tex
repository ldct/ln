\listfiles
\documentclass{article}

\usepackage{amsmath}
\usepackage{amssymb}
\usepackage{tikz-cd}

\usepackage[a4paper,margin=1in]{geometry}

\title{Catsters}
\date{}

\begin{document}
\maketitle

\section{Terminal and Initial objects}

Let $C$ be a category. A terminal object in $C$ is an object $T \in Obj(C)$ such that $\forall X \in Obj(C) \exists ! $ morphism $X \to T$.

Example: one-element sets are terminal in Set.

Example: the trivial group is terminal in Grp.

Example: any one-point space is terminal in Top.

It is pointless to think about ``how many trivial groups there are" since they are all isomorphic (as groups), and the isomorphism is canonical.

Lemma: terminal objects are unique up to unique isomorphism.

Non-examples:

\begin{tikzcd}
\cdot \arrow[r, shift left] \arrow[r, shift right] & \cdot
\end{tikzcd}

\begin{tikzcd}
\cdot \arrow[r] & \cdot & \cdot \arrow[r] & \cdot
\end{tikzcd}

\begin{tikzcd}
\cdot \arrow[r] & \cdot \arrow[r] & \cdot \arrow[r] & \cdot \arrow[r] & \ldots
\end{tikzcd}

Fields (assuming $0 \ne 1$ in a field)

\subsection{Initial 1}

Let $C$ be a category. An initial object in $C$ is an object $I \in Obj(C)$ such that $\forall X \in Obj(C) \exists ! $ morphism $I \to X$.

An initial object in $C$ is a terminal object in $C^{op}$.

Example: the empty set is initial in Set.

Example: the one-point space is initial in Top.

Example: the trivial group is initial in Grp.

The trivial group is initial and terminal in Grp. We call such things null objects. Other examples of categories with null objects: pointed sets, based spaces.

In Cat, the empty category is initial and a category with one object and one morphism is terminal.

In Field, there is no initial object.

\section{Products and Coproducts}

The cartesian product of $\{1, 2\}$ and $\{3, 4\}$ is any four element set, but the set must come with projection maps to $\{1, 2\}$ and $\{3, 4\}$.

Examples: Product topology, group product, sum of vector spaces, product of categories, maximums in a poset.

Coproduct in group: you can try taking the disjoint union, but this is not a group. Hence you generate something freely. Free group. (similar intuition for vector space).

Example: coproduct in based topologies is disjoint union, but where the base points are identified.

Notations: $(f, g)$ vs $f \times g$:

\begin{tikzcd}
 & V \arrow[ldd, "f"', bend right] \arrow[rdd, "g", bend left] \arrow[d, "{(f, g)}", dotted] &  \\
 & A \times B \arrow[ld] \arrow[rd] &  \\
A &  & B
\end{tikzcd}

\begin{tikzcd}
 & X \times Y \arrow[ld] \arrow[rd] \arrow[dd, "f \times g", dotted] &  \\
X \arrow[dd, "f"'] &  & Y \arrow[dd, "g"] \\
 & A \times B \arrow[ld] \arrow[rd] &  \\
A &  & B
\end{tikzcd}

\begin{tikzcd}
 & A \arrow[ldd, bend right] \arrow[rdd, bend left] \arrow[d, "\Delta", dotted] &  \\
 & A \times A \arrow[ld] \arrow[rd] &  \\
A &  & A
\end{tikzcd}

\begin{tikzcd}
V \arrow[d, "\Delta"] \arrow[dd, "{(f, g)}"', bend right=49] \\
V \times V \arrow[d, "f \times g"] \\
A \times B
\end{tikzcd}

Let $1$ be a terminal object. Then $X \times 1 = X$.

\section{Pullbacks and Pushouts}

TBD

\end{document}
