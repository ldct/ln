\listfiles
\documentclass{article}

\usepackage{amsmath}
\usepackage{amssymb}
\usepackage{mathtools}
\usepackage{listings}

\DeclarePairedDelimiter\floor{\lfloor}{\rfloor}
\DeclarePairedDelimiter\ceil{\lceil}{\rceil}
\DeclareMathOperator{\cl}{cl}
\DeclareMathOperator{\E}{E}
\def\Z{\mathbb{Z}}
\def\N{\mathbb{N}}
\def\R{\mathbb{R}}
\def\Q{\mathbb{Q}}
\def\K{\mathbb{K}}
\def\T{\mathbb{T}}
\def\B{\mathcal{B}}
\def\XX{\mathfrak{X}}
\def\YY{\mathfrak{Y}}
\def\AA{\mathfrak{A}}
\def\ZZ{\mathfrak{Z}}
\def\BB{\mathcal{B}}
\def\UU{\mathcal{U}}
\def\MM{\mathcal{M}}
\def\M{\mathfrak{M}}
\def\l{\lambda}
\def\L{\Lambda}
\def\<{\langle}
\def\>{\rangle}
\def\f12{\frac{1}{2}}

\usepackage[a4paper,margin=1in]{geometry}

\setlength{\parindent}{0cm}
\setlength{\parskip}{1em}

\title{Manfrino, Ortega, Delgato Inequalities A Mathematical Olympiad Approach}
\date{}

\begin{document}
\maketitle

\section*{1.25 (Difference of AM and GM)}

Let $p = \sqrt{a}, q = \sqrt{b}$. We have

\begin{align*}
p \le \f12(p+q) \le q \\
\f12 \frac{p+q}{p} \le 1 \le \f12 \frac{p+q}{q} \\
\f12 \frac{p^2 - q^2}{p} \le p-q \le \f12 \frac{p^2 - q^2}{q} \\
\frac{1}{4} \frac{p^2 - q^2}{p^2} \le (p-q)^2 \le \frac{1}{4} \frac{p^2 - q^2}{q^2}
\end{align*}

(Can also be proven by direct computation)

Lesson: AM-GM can be factorized

\section*{1.33 $x^4 + y^4 + 8 \ge 8xy$}

Looks like a special case of 4-term AMGM, $x^4 + y^4 + p^4 + q^4 \ge 4xypq$. Comparing coefficients, we have $p^4 + q^4 = 8, pq = 2$, hence $p = q = \sqrt 2$

\section*{1.38 $a > 1 \implies a^n-1 > n (a^{\frac{n+1}{2}} - a^{\frac{n-1}{2}})$}

\begin{align*}
a^n - 1 &= (a-1) (1 + a + a^2 + \ldots a^{n-1}) \\
&\ge (a-1) (a ^ {1 + 2 + \ldots n-1})^{1/n} \cdot n \\
&= n (a-1) a^\frac{n-1}{2} \\
&= RHS
\end{align*}

\section*{1.39 $(1+a)(1+b)(1+c) \implies 8 \implies abc \le 1$}

$ 1 = \frac{1+a}{2} \frac{1+b}{2} \frac{1+c}{2} \ge \sqrt{abc}$. Now square.

\section*{1.40 $\frac{a^3}{b} + \frac{b^3}{c} + \frac{c^3}{a} \ge ab + bc + ca$}

By cyclic (see section), since $(3, -1, 0) > (1, 1, 0)$. Can also expand to get rid of the -1.

\section*{1.41 $a^2b^2 + b^2c^2 + c^2a^2 \ge abc(a+b+c)$}

By muirhead, since $(2, 2, 0) > (2, 1, 1)$.

\section*{1.51 $a + b + c = 1 \implies \left(\frac{1}{a}+1\right) \left(\frac{1}{b}+1\right) \left(\frac{1}{c}+1\right) \ge 64$}

This is equivalent to $(1+a)(1+b)(1+c) \ge 64abc$. Using $1 = a+b+c$, $(2a + b + c)(a + 2b + c)(a + b + 2c) \ge 64abc$. Apply 4-term AMGM, e.g. the first factor is $\ge (a^2bc)^\frac{1}{4}$.

Note the starting inequality is sharp when $a=b=c$. 3-term doesn't work since it is not sharp in that case.

\section*{1.52 $a + b + c = 1 \implies \left(\frac{1}{a}-1\right) \left(\frac{1}{b}-1\right) \left(\frac{1}{c}-1\right) \ge 8$}

Equivalent to $(b+c)(a+c)(a+b) \ge 8abc$ which holds by AMGM on each factor.

\section*{1.53}

Unsolved...

\section*{1.54}

Let $p = \frac{1}{1+a}$. Then the inequality is equivalent to 1.52.

\section*{1.55}

Equivalent to $HM(a, b) + HM(b, c) + HM(a, c) \le 3*AM(a, b, c)$. Rewrite the RHS as $AM(a, b) + AM(b, c) + AM(a, c)$.

Same strategy used to show that $3*AM(x, y, z) \ge GM(x, y) + GM(y, z) + GM(x, z)$.

\section*{1.56}

\section*{1.57}

\section*{1.58 $x^4 + y^4 + z^2 \ge \sqrt{8} xyz$}

LHS = $x^4 + y^4 + \frac{z^2}{2} + \frac{z^2}{2}$. Now apply 4-term AMGM.

\section*{1.59}

Using the substitution $x = 1+p$ we have $(1+p)^2p + (1+q)^2q \ge 8pq$. We have $(1 + p)^2 \ge 4p$ by AMGM. Then $4(p^2 + q^2) \ge 8pq$ which holds by AMGM.

We know we first have to apply AMGM to the $1+p$ term by dimensional analysis.

Good example of weighted AMGM

\section*{Cyclic $(1, 0, 0) > (p, q, 0)$}

Let $a, b, c > 0, p+q=1$. Define $(r, s, t) = \sum_{cyc} a^r b^s c^t$. Then $(1, 0, 0) > (p, q, 0)$.

Proof: we have $pa + qb \ge a^p b^q, pb + qc \ge b^p + c^q, pc + qa \ge c^p a^q$ by weighted AMGM. Summing them gives $(p+q)(a + b + c) \ge (p, q, 0)$.

\end{document}
