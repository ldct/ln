\listfiles
\documentclass{article}

\usepackage{amsmath}
\usepackage{amssymb}
\usepackage{mathtools}

\DeclarePairedDelimiter\floor{\lfloor}{\rfloor}
\DeclareMathOperator{\Hom}{Hom}

\usepackage[a4paper,margin=1in]{geometry}

\setlength{\parindent}{0cm}
\setlength{\parskip}{1em}


\title{CTCS}
\date{}


\begin{document}
\maketitle


\section*{Sets}

\subsection*{1}

Let $h: W \rightarrow S$ be a function. Define a function $\Hom(h, T): \Hom(S, T) \to \Hom(W, T)$ by $\Hom(h, T)(g) = g \circ h$. Show that if $T$ has at least 2 elements, then $h$ is surjective iif $\Hom(h, T)$ is injective.

$\implies$: Let $h$ be surjective. We wish to show that $\Hom(h, T)$ is injective. Suppose $\Hom(h, T)(f) = \Hom(h, T)(g)$. Then

\begin{align}
\Hom(h, T)(f) &= \Hom(h, T)(g) \\
f \circ h &= g \circ h
\end{align}

Let $y \in S$ be arbitrary. Since $h$ is onto $S$, there exists $x$ such that $y = h(x)$. Then

\begin{align}
f(h(x)) &= g(h(x)) \\
f(y) &= g(y)
\end{align}

Since $y$ was arbitrary, $f = g$. Hence $\Hom(h, T)$ is injective.

$\impliedby$: Let $h$ be not surjective. We wish to show that $\Hom(h, T)$ is not injective. Since $h$ is not surjective there exists $y \in S$ such that $y \ne h(x)$ for all $x$. Let $f$ and $g$ be functions both from XXX that agree on all values in their domain except that $f(y) \ne g(y)$. Note that they are different functions. However $\Hom(h, T)(f) = \Hom(h, T)(g)$ because $g \circ h = f \circ h$. Hence $\Hom(h, T)$ is not injective.

\subsection*{2a}

Show that the mapping that takes a pair $(f: X \to S, g: X \to T)$ of functions to the function $<f, g>: X \to S \times T$ defined by $<f, g>(x) = <f(x), g(x)>$ is a bijection from $\Hom(X, S) \times \Hom(X, T)$ to $\Hom(X, S \times T)$.

Surjective: we show that the range of the mapping is equal to the codomain, $\Hom(X, S \times T)$. Let $h \in \Hom(X, S \times T)$ be given. Then $h: X \to S \times T$. We construct $f$ and $g$ as follows. Let $x$ be arbitrary. Then $h(x) \in S \times T$, so $h(x) = (a, b)$. Then let $f(x) = a, g(x) = b$.

Injective: Suppose $<f, g> = <h, j>$. Then for all $x$ we have

\begin{align*}
<f, g>(x) &= <h, j>(x) \\
(f(x), g(x)) &= (h(x), j(x)) \\
f(x) &= h(x) \\
g(x) &= j(x)
\end{align*}

Since this is true of all $x$ we have $f = h, g = j$

\subsection*{2b}

If you set $X = S \times T$ in (a) what does $id_{S \times T}$ correspond to under the bijection?

\subsection*{3a}

Let $S$ and $T$ be disjoint sets. Let $V$ be a set. Let $\phi: \Hom(S, V) \times \Hom(T, V) \to \Hom(S \cup T, V)$ be the mapping that takes a pair $(f: S \to V, g: T \to V)$ to the function $<f|g>: S \cup T \to V$ defined by $<f|g>(x) = f(x) if x \in S, g(x) if x \in T$. Show that $\phi$ is a bijection.

\subsection*{3b}

If you set $V = S \cup T$ in (a), what is $\phi(id_{S \cup T})$?

\subsection*{4a}

If $P(C)$ denotes the powerset of all subsets of $C$, then $Rel(A, B) = P(A \times b)$ denotes the set of relations from $A$ to $B$. Let $\phi: Rel(A, B) \to \Hom(A, P(B))$ be defined by $\phi(\alpha)(a) = \{b \in B | (a, b) \in \alpha \}$. Show that $\phi$ is a bijection.

\subsection*{4b}

Let $A = B$. What corresponds to $\Delta_A$ under this bijection?

\subsection*{4c}

If we let $A = P(B)$ then $\phi^{-1}: \Hom(P(B), P(B)) \to Rel(P(B), B)$. What is $\phi^{-1}(id_{P(B)})$?

\end{document}
